\section{Strahlungsprozesse der Erde}
% zentrale Frage: WAS BESTIMMT DAS KLIMA

% Um den Klimawandel zu Verstehen, muss erst einmal geklärt werden wie überhaupt ein Klima auf der Erde erreicht werden kann


% Die Grundlage für das Leben auf der Erde bildet zum einen die Sonne, die durch ihre Energie für eine gewisses Wärme bei uns sorgt.
% Allerdings ist der entschiedende Punkt wieso bei uns nicht solche Temepraturen herrschen wie auf unseren nachbarplaneten Venus (464 Grad Celsius auf der Oberfläche) oder Mars (im Mittel ca -55 Grad Celsius) ist die Atmosphäre!

% Sie sorgt dafür, dass eine gewisse Menge der Sonnenenergie auf der Erde verbleibt und somit für lebenfreundliche Temperaturen.

\begin{frame}
  \frametitle{Strahlungsprozesse der Erde}
  %TODO
\end{frame}

\begin{frame}
	\frametitle{Aufbau der Atmosphäre}
	\begin{figure}
		\centering
		\includegraphics{bilder/atmosphaere-stockwerkaufbau_bildungsserver_hh.jpg}
		\caption{Schematischer Aufbau der Atmosphäre, Quelle: Bildungsserver Hamburg} % Zu beachten ist dabei, dass die atmosphärischen Schiten nicht überall gleich groß sind zB geht die Troposphäre im allg bis etwa 10-12km Höhe, in den Tropen aber bis etwa 17-18km Höhe und an den Polen in nur 8 km Höhe	
	\end{figure}
\end{frame}

%Übergang: Ohne die Atmosphäre hätten wir auf der Erde eine durchschnittliche Temperatur von -18 Grad, durch die Atmosphäre und die atmosphärischen Gase ist es aber im Mittel etwa etwa 35 Grad wärmer -> 15 Grad

\begin{frame}
  \frametitle{Strahlungshaushalt}
  
  \begin{figure}
  	\centering
  	\includegraphics[width=0.8\linewidth]{bilder/Energiebilanz_der_Erde_NASA.png}
  	\caption{Strahlungshaushalt der Erde, Quelle: NASA nach Kiehl und Trenberth 1997, Übersetzung: S4F} % Die Abbildung ist in den Folien der S4F Vertiefung zum Klimawandel (S.21) aber dort leider nur spärlich referenziert. Annahme: Jemand der S4F hat dieses Bild Übersetzt und dem Abbildungs-Pool hinzugefügt.
  \end{figure}
  
  %Abbildung für Strahlungshaushalt einfügen z.B: vom Bildungsserver Hamburg https://bildungsserver.hamburg.de/atmosphaere-und-treibhauseffekt/2069644/atmosphaere-strahlungshaushalt-artikel/ 
  % bzw. vom den S4F materialien: http://files.scientists4future.org/Themen/2.%20Klimawandel/Als%20PDFs/S4F-03%20Klima%20Vertiefung%202020-01-31.pdf
  % oder vom IPCC AR5 Figure 2.11
\end{frame}

% Fokus auf den Treibhauseffekt, d.h. vorherige Abbildung reduziert um ein paar Energieflüsse  (mitte)
% zB Sonneneinstrahlung- Atmosphärische Reflexion = 235 W/m2
% Die Abbildung zeigt zum einen die Temperatur auf der Erde ohne die Strahlungsbilanz der Erde(links) außerdem nochmal den atmosphärischen Aufbau (rechts)
% In der Mitte ist die Auswirkung der Treibhausgase zu sehen

\begin{frame}
  \frametitle{Treibhauseffekt}
  \begin{figure}
  	\centering
  	\includegraphics[width=0.8\linewidth]{bilder/Treibhauseffekt_welt_der_physik.png}
  	\caption{Natürlicher Treibhauseffekt, Quelle: Welt der Physik}
  \end{figure}
\end{frame}

\begin{frame}
	\frametitle{Treibhausgase}
	\begin{itemize}		
		\item  {\color{red}{Wasserdampf $H_2O$}} -  großer Teil des natürlicher Treibhauseffekt % Als Brücke von der vorherigen Folie, 60 % des natürlichen Treibhasueffekts lassen sich auf das Vorkommen von Wasserdampf in der Atmosphäre zurückführen
		\item  {\color{red}{Kohlenstoffdioxid $CO_2$}} - entsteht u.A. bei der Nutzung fossiler Brennstoffe (Öl, Kohle, Gas) $<$ $\rightarrow$ auf den Menschen zurückzuführen  
		\item  {\color{red}{Methan $CH_4$}} - entsteht u.a. bei der Zersetzung von organischem Material
		\item[$\rightarrow$] $>$ 60\,\% der globalen Methan-Emissionen sind auf menschliche Aktivitäten zurückzuführen %ClimateChange Kurs der Uni Helsinki Chapter 1.3.3
		\item  {\color{red}{Distickstoffmonoxid $N_2O$ (Lachgas)}} - entsteht beim Abbau von Düngemitteln in der Erde
		\item  {\color{red}{Ozon $O_3$}} - entsteht bei Reaktion von Auto-Abgasen in der Luft im Sonnenlicht
		\item {\color{red}{fluorierte Treibhausgase (F-Gase)}}
		\item[$\rightarrow$] kommen nicht natürlich vor, sehr treibhauswirksam, verweilen z.T. sehr lange in der Atmosphäre, % bis zu 10.000 Jahre
		aber kommen nur in sehr geringer Konzentration vor
		
		% lfu bayern
		% Zu den F-Gasen zählen die voll halogenierten Fluorkohlenwasserstoffe (FKW, englisch: PFC), die teilhalogenierten Fluorkohlenwasserstoffe (HFKW, englisch: HFC), Schwefelhexafluorid (SF6) und Stickstofftrifluorid (NF3).
		% F-Gase kommen in der Natur nicht vor; aber befinden sich u.a. in Gefriertruhen, Klimaanlagen, Feuerlöschern und Dämmstoffen.	
	\end{itemize}
\end{frame}

\begin{frame}
	\frametitle{Treibhausgase} % TODO: Folie evtl. rausnehmen, da zu unübersichtlich
	
	% TODO: Ist das zu unübersichtlich? Der Fokus soll auf der Verweildauer und dem EInfluss auf den Treibhauseffekt liegen --> evtl Informationen entfernen
	
	
	% Werte für CO2, CH4, und N2O von WMO GREENHOUSE GAS BULLETIN No. 15 | 25 November 2019
	% Tabelle von: https://wiki.bildungsserver.de/klimawandel/index.php/Treibhausgase#cite_note-4
	% RF bom Annual Greenhouse Gas index 2018
	
	% ppm steht für parts per million
	\begin{tabular}{p{2.5cm}|p{3cm}|p{3cm}|p{2cm}|p{2cm}}
		Treibhausgas & Konzentration in der Atmosphäre (2018) & mittlerer jährlicher absoluter Anstieg (2008-2018) & mittlere Verweil-dauer in der Atmosphäre [Jahre] & Strahlungs-antrieb (Radiative Forcing) \\ \\ 
		\hline 
		Kohlenstoff-dioxid ($CO_2$) & (407,8 $\pm$ 0,1) $ppm$ & 2,26 $ppm$\,$y^{-1}$ &  30-1000 & $\approx$ \SI{2}{\watt\per\square\metre} \\ 
		\hline 
		Methan ($CH_4$) & (1869 $\pm$ 2) $ppb$ & 7,1 $ppb$\,$y^{-1}$ &9,1 & $\approx $ \SI{0,5}{\watt\per\square\metre}  \\ 
		\hline 
		Lachgas ($N_2O$) & (331,1 $\pm$ 0,1) $ppb$ & 0,95 $ppm$\,$y^{-1}$ & 131 & $\approx $ \SI{0,2}{\watt\per\square\metre}  \\ 
	\end{tabular} 
	Es gibt noch weitere Treibhausgase, die hier aufgelisteten sind die wichtigsten für den anthropogenen Treibhauseffekt.
	
\end{frame}


\begin{frame}
	\frametitle{Treibhausgase}
	\begin{figure}
		\centering
		\includegraphics[width=0.6\linewidth]{bilder/Treibhausgase0-aktuell_bildungsserver_hh.jpg}
		\caption{Treibhausgaskonzentration in der Atmosphäre, Quelle: Bildungsserver Hamburg}
		%Abbildung wie in S.Ranmstorf S.33 -> http://www.pik-potsdam.de/~stefan/Publications/Book_chapters/Der_Klimawandel_Kapitel2.pdf oder direkt im IPCC 2013: https://www.ipcc.ch/report/ar5/wg1/observations-atmosphere-and-surface/ Fig 2.1-2.3
	\end{figure}
\end{frame}



\begin{frame}
	\frametitle{Treibhausgase - Klimawirksamkeit}
	\begin{figure}
		\begin{figure}
			\begin{columns}
				\column{0.6\linewidth}
				\includegraphics[width=0.9\linewidth]{bilder/radiative_forcing_CCnow_mooc.jpg}
				\column{0.3\linewidth}
				\caption{Strahlungsantrieb der Treibhausgase zwischen 1750 und 2011, Quelle: Climate.now MOOC, abgewandelt von IPCC 2014 Kapitel 8}
			\end{columns}
		\end{figure}
	\end{figure}
\begin{itemize}
	\item Die Änderungen des Strahlungsantriebs sind durch den Menschen verursacht (ersichtlich durch den betrachten Zeitraum)
	\item $CO_2$, $CH_4$ und $N_2O$ zählen zu den langlebigen Treibhausgasen und sind über den Globus verteilt.
	\item Der Grad des wissenschaftlichen Verständnis über diese ist hoch!  
	% O3 ist kontinental bis global verbereitet und das Verständnis ist Mittel
	% CH4 ist global verbreitet jedoch ist das Versändnis über den Strahlungsantrieb gering.
	% Das Verständnis sinkt nach Einträgen und daher besonders für die verringernden RF-Faktoren wie Albedo ungewiss. Die Verringerung könnte auch deutlich niedriger liegen, was bedeutet, dass der von den Menschen verursachte Strahlungsantireb sogar noch stärker ist. - Da durch die Erderwärmung und Rußablagerungen auf dem Eis der Albedo-Effekt geschmälert wird, ist das sogar wahrscheinlich.
	% Ergänzungen aus der Abbildung: https://wiki.bildungsserver.de/klimawandel/index.php/Strahlungsantrieb
\end{itemize}
\end{frame}


\begin{frame}
	\frametitle{$CO_2$-Äquivalent}
	
	\begin{block}{Strahlungsantrieb / Radiative Forcing / RF}
		Strahlungsenergie pro Sekunde und pro Quadratmeter, die durch die Tropopause hindurch kommt \\
		Einheit: %\SI{}{\joule\per\second\per\square\metre} bzw. %\SI{}{\watt\per\square\meter}
	\end{block}
	
	\begin{block}{$CO_2$-Äquivalent}
		Integral des RF eines Treibhausgases über einen bestimmten Zeitraum (meist 100 Jahre) im Verhältnis zu dem von $CO_2$
		
%		Beitrag eines Treibhausgases zum Treibhauseffekt über 100 Jahre gemessen am Beitrag des $CO_2$
	\end{block}


	\color{gray}\rule{\linewidth}{1pt}
	
	\color{black}

	2018 lag der Wert an $CO_2$-Äquivalenten in der Atmosphäre bei 496 ppm (RF: 3,101)\\
	\textit{zum Vergleich: } 1990 waren es noch 417 ppm (RF: 2,165)\\
	$\rightarrow$ \color{red}{Zuwachs des Strahlungsantriebs um 43 \% seit 1990}
\end{frame}

\begin{frame}
	\frametitle{Klimawirksamkeit der Treibhausgase}
	\begin{itemize}
		\item Methan ($CH_4$) ist ca. 30-mal klimawirksamer als $CO_2$ 
		\item Lachgas ($N_2O$) ist 265-mal klimawirksamer als $CO_2$
		% das heißt selbst die deutlich kleinere Menge von ihnen in der Atmosphäre (ppb anstatt ppm) hat einen deutlichen Einfluss
	\end{itemize}
%(Quelle: IPCC 2014, Kapitel 8, Tabelle 8.7) %Treibhauswirksamkeit über 100 Jahre ohne Einbezug der Feedbacks - GWP [..] with and without inclusion of climate–carbon feedbacks (cc fb) in response to emissions of the indicated non-CO2 gases

%TODO Abbildung evtl vereinfachen oder bessere Abbildung finden! Einfluss von SO_2 noch nicht betrachtet und generell zu voiele Infos über andere Gase enthalten
% SO_2 könnte Debatte eröffnen, die wir an der Stelle noch nicht führen können
% z.B. Abbildung über Verweildauer in der Atmosphäre IPCC 2014 Kapitel 8 Anhang 8.A
	\begin{columns}
	\column{0.5\linewidth}
		\begin{figure}
			\centering
			\includegraphics[width=\linewidth]{bilder/IPCC_GWP_anthropogenic_emissions.jpg}
			\caption{Globale von Menschen verursachte Emissionen sortiert nach Klimawirksamkeit (GWP), Quelle: IPCC 2014, Kapitel 8}
		\end{figure}
		\column{0.5\linewidth}
		$\rightarrow$ $CO_2$, $CH_4$ und $N_2O$ haben über die Zeit die höchste Klimawirksamkeit\\
		$\rightarrow$ $CO_2$ verliert über den Zeitraum von 100 Jahren kaum an Klimawirksamkeit\\
%		auch Methan bleibt vergleichsweise lange klimawirksam\\
	\end{columns}
\end{frame}

\begin{frame}
	\frametitle{Eis-Albedo-Rückkopplung}%Rahmstorf, Schellnhuber - Der Klimawandel S. 14
	 Reflexion der ankommenden Sonneneinstrahlung durch Eismassen
	 
	 \begin{columns}[T] % align columns
	 	\begin{column}{.48\textwidth}
	 		\centering
	 		\textbf{Abkühlung}\\
	 		\color{blue}\rule{\linewidth}{4pt}
	 		\color{black} 
	 		\color<2->{gray}
	 		je mehr Eismassen\\
	 		$\downarrow$\\
	 		desto mehr wird reflektiert/\\
	 		desto weniger wird absorbiert\\
	 		$\downarrow$\\
	 		desto kälter wird es auf dem Planeten\\
	 		$\downarrow$\\
	 		desto weniger Wasserdampf kann die Atmosphäre aufnehmen\\
	 		$\downarrow$\\
	 		desto geringer wird der Treibhauseffekt
	 	\end{column}%
	 	\hfill%
	 	\begin{column}{.48\textwidth}<2->
	 		\centering
	 		\textbf{Aufwärmung}\\
	 		\color{red}\rule{\linewidth}{4pt}
	 		\color{black}
	 		je weniger Eismassen\\
	 		$\downarrow$\\
	 		desto weniger wird reflektiert/\\
	 		desto mehr wird absorbiert\\
	 		$\downarrow$\\
	 		desto wärmer wird es auf dem Planeten\\
	 		$\downarrow$\\
	 		desto mehr Wasserdampf kann die Atmosphäre aufnehmen\\
	 		$\downarrow$\\
	 		desto stärker wird der Treibhauseffekt
	 	\end{column}%
	 \end{columns}
	 

\end{frame}

% "Runterscrollen" / "Zoom" in einer Abbildung von den Atmosphärischen Schichten auf die Erde
