\section{Strahlungsprozesse der Erde}
% zentrale Frage: WAS BESTIMMT DAS KLIMA

% Um den Klimawandel zu Verstehen, muss erst einmal geklärt werden wie überhaupt ein Klima auf der Erde erreicht werden kann


% Die Grundlage für das Leben auf der Erde bildet zum einen die Sonne, die durch ihre Energie für eine gewisses Wärme bei uns sorgt.
% Allerdings ist der entschiedende Punkt wieso bei uns nicht solche Temepraturen herrschen wie auf unseren nachbarplaneten Venus (464 Grad Celsius auf der Oberfläche) oder Mars (im Mittel ca -55 Grad Celsius) ist die Atmosphäre!

% Sie sorgt dafür, dass eine gewisse Menge der Sonnenenergie auf der Erde verbleibt und somit für lebenfreundliche Temperaturen.

\begin{frame}
  \frametitle{Strahlungsprozesse der Erde}
  % 
\end{frame}

\begin{frame}
  \frametitle{Strahlungshaushalt}
  
  \begin{figure}
  	\caption{Strahlungshaushalt der Erde}
  \end{figure}
  
  %Abbildung für Strahlungshaushalt einfügen z.B: vom Bildungsserver Hamburg https://bildungsserver.hamburg.de/atmosphaere-und-treibhauseffekt/2069644/atmosphaere-strahlungshaushalt-artikel/ 
  % bzw. vom den S4F materialien: http://files.scientists4future.org/Themen/2.%20Klimawandel/Als%20PDFs/S4F-03%20Klima%20Vertiefung%202020-01-31.pdf
  % oder vom IPCC AR5 Figure 2.11
\end{frame}

\begin{frame}
  \frametitle{Treibhauseffekt}
\end{frame}

\begin{frame}
	\frametitle{Treibhausgase}
	\begin{itemize}		
		\item  {\color{red}{Wasserdampf $H_2O$}} -  großer Teil des natürlicher Treibhauseffekt % Als Brücke von der vorherigen Folie, 60 % des natürlichen Treibhasueffekts lassen sich auf das Vorkommen von Wasserdampf in der Atmosphäre zurückführen
		\item  {\color{red}{Kohlenstoffdioxid $CO_2$}} -  entsteht bei der Nutzung fossiler Brennstoffe (Öl, Kohle, Gas) $<$ $\rightarrow$ auf den Menschen zurückzuführen  
		\item  {\color{red}{Methan $CH_4$}} - entsteht u.a. bei der Zersetzung von organischem Material
		\item[$\rightarrow$] $>$ 60 \% der globalen Methan-Emissionen sind auf menschliche Aktivitäten zurückzuführen %ClimateChange Kurs der Uni Helsinki Chapter 1.3.3
		\item  {\color{red}{Distickstoffmonoxid $N_2O$ (Lachgas)}} - entsteht beim Abbau von Düngemitteln in der Erde
		\item  {\color{red}{Ozon $O_3$}} - entsteht bei Reaktion von Auto-Abgasen in der Luft im Sonnenlicht
		\item {\color{red}{fluorierte Treibhausgase (F-Gase)}}
		\item[$\rightarrow$] kommen nicht natürlich vor, sehr treibhauswirksam, verweilen z.T. sehr lange in der Atmosphäre, % bis zu 10.000 Jahre
		aber kommen nur in sehr geringer Konzentration vor
		
		% lfu bayern
		% Zu den F-Gasen zählen die voll halogenierten Fluorkohlenwasserstoffe (FKW, englisch: PFC), die teilhalogenierten Fluorkohlenwasserstoffe (HFKW, englisch: HFC), Schwefelhexafluorid (SF6) und Stickstofftrifluorid (NF3).
		% F-Gase kommen in der Natur nicht vor; aber befinden sich u.a. in Gefriertruhen, Klimaanlagen, Feuerlöschern und Dämmstoffen.	
	\end{itemize}
\end{frame}

\begin{frame}
	\frametitle{Treibhausgase}
	% TODO: Ist das zu unübersichtlich? Der Fokus soll auf der Verweildauer und dem EInfluss auf den Treibhauseffekt liegen --> evtl Informationen entfernen
	
	
	% Werte für CO2, CH4, und N2O von WMO GREENHOUSE GAS BULLETIN No. 15 | 25 November 2019
	% Tabelle von: https://wiki.bildungsserver.de/klimawandel/index.php/Treibhausgase#cite_note-4
	% RF bom Annual Greenhouse Gas index 2018
	
	% ppm steht für parts per million
	\begin{tabular}{p{2.5cm}|p{3cm}|p{3cm}|p{2cm}|p{2cm}}
		Treibhausgas & Konzentration in der Atmosphäre (2018) & mittlerer jährlicher absoluter Anstieg (2008-2018) & mittlere Verweil-dauer in der Atmosphäre [Jahre] & Strahlungs-antrieb (Radiative Forcing) [$W/m^2$] \\ 
		\hline 
		Kohlenstoff-dioxid ($CO_2$) & 407,8 $\pm$ 0,1 $ppm$ & 2,26 $ppm$ $y^{-1}$ &  30-1000 & 2,044 \\ 
		\hline 
		Methan ($CH_4$) & 1869 $\pm$ 2 $ppb$ & 7,1 $ppb$ $y^{-1}$ &9,1 & 0,512  \\ 
		\hline 
		Lachgas ($N_2O$) & 331,1 $\pm$ 0,1 $ppb$ & 0,95 $ppm$ $y^{-1}$ & 131 & 0,199  \\ 
	\end{tabular} 
	\color<2->{black} Es gibt noch weitere Treibhausgase, die hier aufgelisteten sind die wichtigsten für den Treibhauseffekt.
	
\end{frame}

\begin{frame}
	\frametitle{Treibhausgase}
	\begin{figure}
		\caption{Anstieg der Treibhausgaskonzentration}
		%Abbildung wie in S.Ranmstorf S.33 -> http://www.pik-potsdam.de/~stefan/Publications/Book_chapters/Der_Klimawandel_Kapitel2.pdf oder direkt im IPCC 2013: https://www.ipcc.ch/report/ar5/wg1/observations-atmosphere-and-surface/ Fig 2.1-2.3
	\end{figure}
\end{frame}

\begin{frame}
	\frametitle{$CO_2$-Äquivalent}
	
	\begin{block}{Strahlungsantrieb / Radiative Forcing / RF}
		Strahlungsenergie pro Sekunde und pro Quadratmeter, die durch die Tropopause hindurchkommt \\
		Einheit: $J/ sm^2$ bzw. $W/m^2$ 
	\end{block}
	
	\begin{block}{$CO_2$-Äquivalent}
		Integral des RF eines Treibhausgases über einen bestimmten Zeitraum (meist 100 Jahre) im Verhältnis zu dem von $CO_2$
		
%		Beitrag eines Treibhausgases zum Treibhauseffekt über 100 Jahre gemessen am Beitrag des $CO_2$
	\end{block}


	\color{gray}\rule{\linewidth}{1pt}
	
	\color{black}

	2018 lag der Wert an $CO_2$-Äquivalenten in der Atmosphäre bei 496 ppm (RF: 3,101)\\
	\textit{zum Vergleich: } 1990 waren es noch 417 ppm (RF: 2,165)\\
	$\rightarrow$ \color{red}{Zuwachs des Strahlungsantriebs um 43 \% seit 1990}
\end{frame}

\begin{frame}
	\frametitle{Eis-Albedo-Rückkopplung}%Rahmstorf, Schellnhuber - Der Klimawandel S. 14
	 Reflexion der ankommenden Sonneneinstrahlung durch Eismassen
	 
	 \begin{columns}[T] % align columns
	 	\begin{column}{.48\textwidth}
	 		\centering
	 		\color{blue}\rule{\linewidth}{4pt}
	 		\color{black} 
	 		\color<2->{gray}
	 		je mehr Eismassen\\
	 		$\downarrow$\\
	 		desto mehr wird reflektiert/\\
	 		desto weniger wird absorbiert\\
	 		$\downarrow$\\
	 		desto kälter wird es auf dem Planeten\\
	 		$\downarrow$\\
	 		desto weniger Wasserdampf kann die Atmosphäre aufnehmen\\
	 		$\downarrow$\\
	 		desto geringer wird der Treibhauseffekt
	 	\end{column}%
	 	\hfill%
	 	\begin{column}{.48\textwidth}<2->
	 		\centering
	 		\color{red}\rule{\linewidth}{4pt}
	 		\color{black}
	 		je weniger Eismassen\\
	 		$\downarrow$\\
	 		desto weniger wird reflektiert/\\
	 		desto mehr wird absorbiert\\
	 		$\downarrow$\\
	 		desto wärmer wird es auf dem Planeten\\
	 		$\downarrow$\\
	 		desto mehr Wasserdampf kann die Atmosphäre aufnehmen\\
	 		$\downarrow$\\
	 		desto stärker wird der Treibhauseffekt
	 	\end{column}%
	 \end{columns}
	 

\end{frame}

% "Runterscrollen" / "Zoom" in einer Abbildung von den Atmosphärischen Schichten auf die Erde