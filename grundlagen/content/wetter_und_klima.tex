\section{Wetter und Klima}

\begin{frame}
	\frametitle{Wetter und Klima}
  \begin{columns}[onlytextwidth]
    \begin{column}[t]{0.49\linewidth}
    \textbf{Wetter}\\
      \textit{\enquote{ist der physikalische Zustand der Atmosphäre an einem bestimmten Ort zu einem \alert{bestimmten Zeitpunkt} oder in einem kurzen Zeitraum.}}
      \begin{itemize}
        \item Gekennzeichnet durch Ist-Werte von Temperatur, Luftfeuchtigkeit, Niederschlag, ...
        \item Ist was wir täglich mit unseren Sinnen erleben
      \end{itemize}
    \end{column}%
    \begin{column}[t]{0.49\linewidth}
      \textbf{Klima}\\
      \textit{\enquote{ist der mittlere Zustand der Atmosphäre an einem bestimmten Ort über einen \alert{längeren Zeitraum.}}}
      \begin{itemize}
        \item Gekennzeichnet durch statistische Mittelwerte (circa 30 Jahre) der selben Größen
        \item Ist entscheidend für die Entwicklung der Ökosysteme
      \end{itemize}
    \end{column}%
  \end{columns}

  \bigskip
  \begin{center}
    \textbf{Kurz gesagt:}\\
    Das (durchschnittliche) Wetter macht das Klima, \\
    das Klima bestimmt das (wahrscheinliche) Wetter.
  \end{center}


  \vfill
	\tiny{Zitate von \url{https://www.umweltbundesamt.de/service/uba-fragen/was-ist-eigentlich-klima}}
\end{frame}
