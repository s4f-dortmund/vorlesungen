\section{Wetter und Klima}

\begin{frame}
  \frametitle{Wetter und Klima}
  \begin{block}{Wetter}  	
  	kurzfristiger Zustand der Atmosphäre (Skala: Minuten bis Tage)
  \end{block}
  \begin{block}{Klima}
  	gemittelte Wetterereignisse und deren Varianz in einer Region über einen längeren Zeitraum (empfohlene Skala: meist 30 Jahre oder mehr) % laut der Weltorganisation für Meteorologie WMO
  \end{block}
% TODO: Wetter vs. Klimadaten Dortmund: zB Osterwetter vs klassisches Niederschlags-Temperatur-Diagramm anfertigen

% TODO: Daten dazu beim dwd oder bei der Stadt Dortmund Stichwort Opendata -Weather bzw. Witterung

%TODO: Campuswetter der TU NICHT nutzbar, da urheberrechtlich geschützt
\end{frame}

\begin{frame}
	\frametitle{Klima}
	\begin{itemize}
		\item auch definiert als Zustand des klimatischen Systems %WMO
		\item Betrachtung verschiedener Parameter wie Temperatur, Niederschlag, Windgeschwindigkeiten % WMO 
		\item betrachtet werden u.a. Mittelwert, Abweichung und Wahrscheinlichkeit des Auftretens extremer Ereignisse z.B. Dürren %M.Latif, Klimawandel und Klimadynamik S. 11
		\item aus dem Griechischen: \textit{klinein} $\equiv$ neigen $\rightarrow$ Neigung der Erdachse
	\end{itemize}

	%TODO: Abbildung für Neigung der Erde (Sommer/Winter auf der Nordhalbkugel) und Position der Sonne
\end{frame}

\begin{frame}
	\frametitle{Klimaänderung}
	\begin{block}{Klimaänderung}		
		erkennbare (messbare) Änderung Klimas über einen gewissen Zeitraum (z.B. 30 Jahre) %M.Latif Klimawandel und Klimadynamik S. 13
		Spürbar durch Änderung atmosphärischer Größen wie der Temperatur
	\end{block}
	
	%TODO: Abbildung der Klimareihe über die letzten 60 Jahre
	%TODO: Durch Verschiebung des betrachteten Zeitraums (sliding Window) werden Änderungen im Mittelwert der Temperatur deutlich - als Animation oder zum durchklicken möglich
	
	\begin{figure}
		\caption{Sliding Window über den Warming Stripes der letzten 60 Jahre}
	\end{figure}
	
\end{frame}

\begin{frame}
	\frametitle{Klimavorhersagen}
	\begin{itemize}
		\item Die statistischen Eigenschaften des Wetters lassen sich längerfristig Vorhersagen.
		\item[$\rightarrow$] Jahreszeiten, Monsunregen, etc.. %TODO: Bedingungen dafür in  Kapitel 3.3 aus M.Latif nachlesen
		\item Atmosphärische Änderungen im Kontext des Erdsystems
		\item [$\rightarrow$] Hydrosphäre, Kryosphäre, Biospähre, Pedosphäre und Lithosphäre % Stichwort Wechselwirkungen
	\end{itemize}
	
\end{frame}


\begin{frame}
	\frametitle{Klima}
	\begin{itemize}
		\item auch definiert als Zustand des klimatischen Systems %WMO
		\item Betrachtung verschiedener Parameter wie Temperatur, Niederschlag, Windgeschwindigkeiten % WMO 
		\item betrachtet werden u.a. Mittelwert, Abweichung und Wahrscheinlichkeit des Auftretens extremer Ereignisse z.B. Dürren %M.Latif, Klimawandel und Klimadynamik S. 11
		\item aus dem Griechischen: \textit{klinein} $\equiv$ neigen $\rightarrow$ Neigung der Erdachse
	\end{itemize}
	
	%TODO: Abbildung für Neigung der Erde (Sommer/Winter auf der Nordhalbkugel) und Position der Sonne
\end{frame}

\begin{frame}
	\frametitle{Klimaänderung}
	\begin{block}{Klimaänderung}		
		erkennbare (messbare) Änderung Klimas über einen gewissen Zeitraum (z.B. 30 Jahre) %M.Latif Klimawandel und Klimadynamik S. 13
		Spürbar durch Änderung atmosphärischer Größen wie der Temperatur
	\end{block}
	
	%TODO: Abbildung der Klimareihe über die letzten 60 Jahre
	%TODO: Durch Verschiebung des betrachteten Zeitraums (sliding Window) werden Änderungen im Mittelwert der Temperatur deutlich - als Animation oder zum durchklicken möglich
	
	\begin{figure}
		\caption{Sliding Window über den Warming Stripes der letzten 60 Jahre}
	\end{figure}
	
\end{frame}

\begin{frame}
	\frametitle{Klimavorhersagen}
	\begin{itemize}
		\item Die statistischen Eigenschaften des Wetters lassen sich längerfristig Vorhersagen.
		\item[$\rightarrow$] Jahreszeiten, Monsunregen, etc.. %TODO: Bedingungen dafür in  Kapitel 3.3 aus M.Latif nachlesen
		\item Atmosphärische Änderungen im Kontext des Erdsystems
		\item [$\rightarrow$] Hydrosphäre, Kryosphäre, Biospähre, Pedosphäre und Lithosphäre % Stichwort Wechselwirkungen
	\end{itemize}
	
\end{frame}

