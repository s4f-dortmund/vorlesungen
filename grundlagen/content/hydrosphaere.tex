\begin{frame}
	\frametitle{Hydrosphäre - Wassermassen}

	\begin{figure}
		\centering
		\includegraphics[trim={1cm 0cm 0cm 3cm}, clip, width=0.55\linewidth]{%
        bilder/climate_components/global_climate_components_hydrosphere.pdf}
		\caption{Etwa 2/3 der Erdoberfläche sind von Wasser bedeckt, somit ist die Hydrosphäre allein durch die Masse ein wichtiger Faktor des Klimasystems.}
	\end{figure}

	\note{
	\begin{itemize}
		\item[] Etwa 2/3 der Erdoberfläche sind von Wasser bedeckt, somit ist die Hydrosphäre allein durch die Masse ein wichtiger Faktor des Klimasystems.
		% Aufteilung von https://www.lenntech.de/faq-wasser-menge.htm, aber auch Wetzel, Robert G. 1983 in "Periphyton of freshwater ecosystems"
    \item[] Der Großteil der Wassermassen ist Salzwasser 97\,\%.
    \item[] Süßwasser (Flüsse und Seen) bildet nur weniger als 1\,\% des globalen Wassers.
    \item[] Eismassen machen ca. 2\,\% des Wassers aus und binden den größten Teil des Süßwassers, da das Salz bei der Eisbildung im Wasser gelößt bleibt.
    \item[] Eis betrachten wir später (Kryosphäre).
	\end{itemize}
	}
\end{frame}


\begin{frame}
	\frametitle{Eigenschaften des Wassers} % M.Latif Klimawandel und Klimadynamik S. 230.5
	\begin{columns}
		\column[t]{0.5\linewidth}
		\begin{itemize}
			\item Wasser ist ein Dipol-Molekül \textcolor{blue}{H$_2$O}
			\begin{itemize}
				\item[$\rightarrow$] Kann wirksam Infrarotstrahlung absorbieren
				\item[$\rightarrow$] Kann viel Wärme aufnehmen bevor es verdampft $\rightarrow$ "Trägheit"
			\end{itemize}
			%Allgemein liegt die größte Dichte des Wassers bei 4 Grad
			\item<2-> Größte Dichte von reinem Wasser bei \SI{4}{°C}
			\begin{itemize}
				\item<2->[$\rightarrow$] Eis schwimmt auf Wasser
			\end{itemize}
			% Im besonderen Fall des Salzwassers liegt die größte Dichte jedoch bei -3.8 Grad
			\item<3->Größte Dichte von \textit{Salz}wasser bei \SI{-3,8}{°C}
			\begin{itemize}
				\item<3-> [] Bei der Eisbildung verbleibt das Salz gelöst im Wasser
				\item<3-> [$\rightarrow$] Salzwasser kann kälter werden als Eis und sinkt in die tieferen Schichten des Ozeans
				\item<3-> [$\rightarrow$] Wärmeres, weniger dichtes Wasser steigt auf
				\item<3-> [] \textbf{\textcolor{blue}{Konvektion}} in den Polarregionen
				\item<3-> [$\rightarrow$] Kohlenstoffsenken
			\end{itemize}
		\end{itemize}
		\column[t]{0.5\linewidth}
		\begin{figure}
			\includegraphics[scale=0.7]{bilder/wasser_molekuel}\\		\caption{Schematische Darstellung eines Wasser Moleküls, Quelle: chemga}
		\end{figure}
	\end{columns}
	\note{
		\begin{itemize}
      \item[] Wasser ist ein Dipol-Molekül
			\item[] Wärmekapazitäten im Vergleich
			\begin{itemize}
				\item[] Wasserstoff (Gas) \SI{14,3}{J\per g\per K}
				\item[] Helium (Gas) \SI{5,2}{J\per g\per K}
				\item[] Ammoniak (Flüssigkeit) \SI{4,7}{J\per g\per K}
				\item[] Lithium (Flüssigkeit) \SI{4,4}{J\per g\per K}
				\item[] Wasser (Flüssigkeit) \SI{4,2}{J\per g\per K}
			\end{itemize}
			\item[] Das Phänomen Konvektion ist nicht auf Wasser beschränkt. Gibt es auch in der Luft oder wird durch Pumpen erzeugt.
			\item[] Die Konvektion ist durch die temperaturbedingte Dichte und den Salzgehalt möglich.
			\item[] Daher ist diese auch unter dem Namen \textit{thermohaline Konvektion} (thermo - Temperatur, halin - Salz) bekannt.
			\item[] Oberflächen nahes, Kohlenstoff reiches Wasser wird abgesenkt $\rightarrow$ Kohlenstoffsenken
		\end{itemize}
	}

% TODO: evtl. Abbildung der Konvektion: Abgabe von Wärme bei Aufnahme von atmosphärischen Gasen, die dann in die Tiefsee gelangen und dort gespeichert werden

%TODO: Erklärung von Senken
\end{frame}

\begin{frame}
	\frametitle{Wasserdynamik} %M. Latif Klimawandel und Klimadynamik S.24
	\begin{figure}
    \centering
    \includegraphics[width=.9\linewidth]{bilder/Thermohaline_Circulation.pdf}
    \caption{Termohaline Zirkulation, Quelle: nach Robert Simmon, NASA}
  \end{figure}

	\note{
		\begin{itemize}
			\item[] Die großen Wassermassen der Ozeane haben eine bestimmte Strömung, z.B. Golfstrom von der Arktis über die Küste Mexikos bis an die Antarktis.
			\item[] Die Stömung ergibt sich vorallem durch die Erdrotation (Corioliskraft).
			\item[] Die oberflächliche Strömung der oberen 100 Meter entsteht durch Wind (und darauf resultierende Reibung) und die Form der Meeresbecken.
			\item[] Die Tiefenströmung wird durch die Konvektion angetrieben. Die Dichte des Wassers spielt dabei eine entschiedende Rolle, da dichteres Wasser nach unten sinkt und leichteres empor steigt.
			\item[] Warme Temperaturen führen zum aufwärmen des Oberflächenwassers und damit zu einer geringeren Dichte.
			\item[] Dadurch werden Oberflächenwasser und Tiefenwasser stark getrennt - und somit strömen die Wassermassen mit unterschiedlicher Geschwindigkeit.
      \item[] Unterscheidung zwischen zwei Zirkulationen
      \begin{itemize}
        \item[Windgetrieben] Oberflächenströmung der Ozeane durch Reibung, Erdrotation (Corioliskraft) und Form der Meeresbecken, eher horizontal
        \item[Dichtegetrieben] Erwärmung, Abkühlung, und Änderung des Salzgehaltes (durch Eisbildung, Verdunstung oder Niederschlag) haben Einfluss auf die Dichte des Wassers, wodurch die Wassermassen zirkulieren, eher vertikal
      \end{itemize}
			\item[] Eine komplette Umwälzung benötigt etwa 1000 Jahre.
		\end{itemize}
	}

\end{frame}


\begin{frame}
	\frametitle{Wirkung des Wassers auf das Klima}
	\begin{columns}[c]
		\column{0.44\linewidth}
			\begin{figure}
				\includegraphics[scale=0.6]{bilder/hydrosphere_energy-in-components.jpg}
				\caption{Energiegehalt der Komponenten des Klimasystems von 1979 bis 2010. Quelle: IPCC 2013, Physical Science Base, Kapitel 3}
			\end{figure}
		\column{0.56\linewidth}
			\begin{itemize}
				\item Ca. 70\,\% der Erdoberfläche ist mit Wasser bedeckt
				\begin{itemize}
					\item[$\rightarrow$] Die \textit{Trägheit} des Wassers ist ein entscheidender Faktor für die Trägheit des Klimas und der Klimaänderungen % M.Latif Klimawandel und Klimadynamik S. 23
				\end{itemize}
				\item Kohlenstoffsenken in der Tiefsee können durch Erwärmen der Ozeane \textit{irgendwann} freigesetzt werden
				\begin{itemize}
					\item[$\rightarrow$] Massiver Anstieg des atmosphärischen CO$_2$
					\item[$\rightarrow$] Verstärkung des Treibhauseffekts
				\end{itemize}
				\item Positive Verstärkung von CO$_2$ und Wasserdampf
				\begin{itemize}
					\item[$\rightarrow$] Eine wärmere Atmosphäre kann mehr Wasserdampf (und CO$_2$) aufnehmen (Eis-Albedo-Rückkopplung)
				\end{itemize}
			\end{itemize}
	\end{columns}
	\note{
		\begin{itemize}
			\item Ein Großteil der Erdoberfläche ist mit Wasser bedeckt, nämlich ca. 70\,\%.
			\item Die Abbildung zeigt, dass ein Großteil des global steigenden Energiegehalts von den Ozeanen und der Tiefsee aufgenommen wurde.
			\item[$\rightarrow$] Dies hat eine erhöhte Temperatur der Wassermassen zur Folge.
			\item Die Trägheit des Wassers ist ein entscheidener Faktor für die Trägheit des Klimas und von Klimaveränderungen.
			\item Eine Veränderung der Wassermengen, -temperatur und -zirkulation wirkt sich so erst mit Verzögerung auf die Klimasysteme aus.
			\item Ein Teil des atmosphärischen $CO_2$ löst sich an der Grenzschicht zwischen Atmosphäre und Hydrosphäre in Wasser, daher speichern Ozeane und die Tiefsee eine große Menge CO$_2$ (Kohlenstoffsenke).
			\item Die Kohlenstoffsenken der Tiefsee könnten durch ein Erwärmen der Ozeane \textit{irgendwann} freigesetzt werden.
			\item[$\rightarrow$] Dies würde in der Folge wiederum zu einem massiven Anstieg des atmosphärischen CO$_2$ und einer Verstärkung des Treibhausgaseffekts führen.
			\item Eine wärmere Atmosphäre kann mehr Wasserdampf und CO$_2$ aufnehmen, wodurch es zu einer positiven Verstärkung des Treibhauseffekts kommt.
		\end{itemize}
	}
\end{frame}

\begin{frame}
	\frametitle{Trägheit des Klimas}
	\begin{figure}
		\centering
		\includegraphics[width=0.8\linewidth]{bilder/zeitskala-klimasystem_world_ocean_review.jpg}
		\caption{Zeitskalen im Klimasystem, Quelle: maribus nach Meinecke und Latif, 1995}
		\label{fig:traegheit}
	\end{figure}

  \begin{itemize}
    \item[$\rightarrow$] Verzögertes Feedback bis zu einem klimawirksamen Ereignis.
    \item[$\rightarrow$] Besonders die Ozeane und Eisschilde benötigen eine sehr lange Zeit, um sich geänderten klimatischen Bedingungen anzupassen.
  \end{itemize}

	\note{
		\begin{itemize}
			\item[] Trägheit des Wassers ist entscheident.
			\item[] Die Wassermassen sind in dieser Abbildung stark vertreten.
			\item[] Die untere Atmosphäre passt sich innerhalb weniger Stunden den Bedingungen der Erdoberfläche an (Temeperatur, Gase, etc.)
			\item[] Die Wassermassen reagieren sehr unterschiedlich.
			\item[] Flüsse, Seen und Oberflächenwasser wärmen sich dabei deutlich schneller auf als die tieferen Ozeanschichten. (Das kennt man vielleicht aus Bade- oder Bergseen - die oberen 50 cm sind angenehm warm und darunter liegt deutlich kälteres Wasser)
			\item[] Besonders unterschiedlich schnell reagiert die Biosphäre.
      \begin{itemize}
        \item[] Graslandschaften können schnell austrocken
        \item[] Wälder dagegen verändern sich über Jahrtausende hinweg.
        \item[] Die Vegetation bestimmt in vielen Fällen auch die Ansiedlung von Lebewesen.
        \item[] Die Änderung der Vegetation kann das Ende des Lebenraums einiger Lebewesen bedeuten, aber auch neue Ansiedluneg bedingen.
      \end{itemize}
			\item[] Eine besonders lange Reaktionszeit haben die Eisschilde der Erde. Auf die Eismassen gehen wir als nächstes ein.
		\end{itemize}
	}
\end{frame}
