\begin{frame}
	\frametitle{Referenzen}
	\small{
	\textbf{Bildungsserver Hamburg,} Hamburger Bildungsserver (HBS), \url{https://bildungsserver.hamburg.de/klimawandel/} \\

	\textbf{IPCC, 2001:} Climate Change 2001: The Scientific Basis. Contribution of Working Group I to the Third Assessment Report of theIntergovernmental Panel on Climate Change[Houghton, J.T., Y. Ding, D.J. Griggs, M. Noguer, P.J. van der Linden, X. Dai, K.Maskell, and C.A. Johnson (eds.)]. Cambridge University Press, Cambridge, United Kingdom and New York, NY, USA, 881pp., \url{https://www.ipcc.ch/report/ar3/wg1/}\\

	\textbf{IPCC 2007:} IPCC, 2007: Climate Change 2007: The Physical Science Basis. Contribution of Working Group I to the Fourth Assessment Report of the Intergovernmental Panel on Climate Change [Solomon, S., D. Qin, M. Manning, Z. Chen, M. Marquis, K.B. Averyt, M. Tignor and H.L. Miller (eds.)]. Cambridge University Press, Cambridge, United Kingdom and New York, NY, USA, 996 pp.,  \url{https://www.ipcc.ch/report/ar4/wg1/}\\

	\textbf{IPCC, 2014:} IPCC, 2013: Climate Change 2013: The Physical Science Basis. Contribution of Working Group I to the Fifth Assessment Report of the Intergovernmental Panel on Climate Change [Stocker, T.F., D. Qin, G.-K. Plattner, M. Tignor, S.K. Allen, J. Boschung, A. Nauels, Y. Xia, V. Bex and P.M. Midgley (eds.)]. Cambridge University Press, Cambridge, United Kingdom and New York, NY, USA, 1535 pp., \url{https://www.ipcc.ch/report/ar5/wg1/}\\

	\textbf{IPCC, 2018:} IPCC, 2018: : Global warming of 1.5°C. An IPCC Special Report on the impacts of global warming of 1.5°C above pre-industrial levels and related global greenhouse gas emission pathways, in the context of strengthening the global response to the threat of climate change, sustainable development, and efforts to eradicate poverty [V. Masson-Delmotte, P. Zhai, H. O. Pörtner, D. Roberts, J. Skea, P.R. Shukla, A. Pirani, W. Moufouma-Okia, C. Péan, R. Pidcock, S. Connors, J. B. R. Matthews, Y. Chen, X. Zhou, M. I. Gomis, E. Lonnoy, T. Maycock, M. Tignor, T. Waterfield (eds.)]. In Press., \url{https://www.ipcc.ch/sr15/}\\
	}
\end{frame}

\begin{frame}
	\frametitle{Referenzen}
	\small{
	\textbf{Mojib Latif}, Klimawandel und Klimadynamik, 2009, UTB GmbH, Auflage: 1, ISBN 978-3-8252-3178-1\\
	\textbf{S4F}, Scientists for Future, \url{https://www.scientists4future.org/}\\
	\textbf{WMO}, World Meteorological Organization, Video zum Kohlenstoffkreislauf: \url{https://www.youtube.com/watch?v=E8Y6L5TI\_94}\\
	\textbf{Wiki1}, Von Mario Sarto - Selbst fotografiert, CC BY-SA 3.0, \url{https://commons.wikimedia.org/w/index.php?curid=1015397}\\
	\textbf{Wiki2}, Von \url{http://resourcescommittee.house.gov/subcommittees/emr/usgsweb/photogallery/images/Coal\%20anthracite\_jpg}, Gemeinfrei, \url{https://commons.wikimedia.org/w/index.php?curid=22263}\\
	\textbf{Wiki3}, Von Topfklao (Christoph Neumüller) at de.wikipedia - Eigenes Werk, Attribution, \url{https://commons.wikimedia.org/w/index.php?curid=4038579}\\
	\textbf{Wiki4} By Benjah-bmm27 - Own work, Public Domain, \url{https://commons.wikimedia.org/w/index.php?curid=940830}\\
	\textbf{Wiki5} Von user And1mu - modified version of, CC BY-SA 4.0, \url{https://commons.wikimedia.org/w/index.php?curid=59505315}\\
	\textbf{Wiki6} Von Stkl - Spectral lines continuous.png, Gemeinfrei, \url{https://commons.wikimedia.org/w/index.php?curid=42405328}\\
	\textbf{Len}, \textit{Tipping elements in the Earth's climate system},
	Timothy M. Lenton, Hermann Held, Elmar Kriegler, Jim W. Hall, Wolfgang Lucht, Stefan Rahmstorf, and Hans Joachim Schellnhuber,
	PNAS February 12, 2008 105 (6) 1786-1793; first published February 7, 2008 \url{https://doi.org/10.1073/pnas.0705414105} \\
	\textbf{UBA} \textit{KIPP-PUNKTE IM KLIMASYSTEM - Welche Gefahren drohen?}, Umweltbundesamt Juli 2008, \url{https://www.umweltbundesamt.de/sites/default/files/medien/publikation/long/3283.pdf} (Sekundärquelle)
	}
\end{frame}

\begin{frame}
	\frametitle{Referenzen}
	\small{
	\textbf{King} King, M.D., Howat, I.M., Candela, S.G. et al. \textit{Dynamic ice loss from the Greenland Ice Sheet driven by sustained glacier retreat}. Commun Earth Environ 1, 1 (2020). \url{https://doi.org/10.1038/s43247-020-0001-2}\\
	\textbf{Global Historical Climate Network}, Menne, M.J., I. Durre, B. Korzeniewski, S. McNeal, K. Thomas, X. Yin, S. Anthony, R. Ray, R.S. Vose, B.E.Gleason, and T.G. Houston, 2012: Global Historical Climatology Network - Daily (GHCN-Daily), Version 3, NOAA National Climatic Data Center. \url{http://doi.org/10.7289/V5D21VHZ}, images of density of weather stations from \url{https://www.ncdc.noaa.gov/ghcn-daily-description}\\
	\textbf{LSBU}, \url{http://www1.lsbu.ac.uk/water/water_phase_diagram.html}\\
	\textbf{chemga}, \url{http://www.chemgapedia.de/vsengine/vlu/vsc/de/ch/8/bc/vlu/chem_grundlagen/wasser.vlu/Page/vsc/de/ch/8/bc/chemische_grundlagen/wasser1.vscml.html}\\
	\textbf{USGS}, United States Geological Survey \url{https://de.wikipedia.org/wiki/Datei:Tectonic_plates_de.svg}
	}
\end{frame}
