\begin{frame}
	\frametitle{Aktuelle Nachrichten}
		\begin{figure}
		\centering
		\includegraphics[width=0.9\linewidth]{bilder/CurrentSituation}
		\end{figure}

		\note{
			Wetterereignisse der letzten drei Jahre
			\begin{itemize}
				\item[2017] Hochwasserkatastrophe in Sierra Leone durch bis du dreifach höheren Regenfall als üblich, keine Unwetterwarnung der Regierung, unkontrollierte Entwaldung, Müllverstopfte Kanalisation, etc.
				\item[2018] Überschwemmungen in Nigeria und vielen anderen afrikanischen Ländern, mehrere Millionen Menschen obdachlos
				\item[2019] Buschbrände in Kalifornien, mehr als \SI{1000}{km\squared} (etwa drei mal die Fläche von Dortmund) verbrannt, gehört zu den größten 5 Buschbränden. 4 der größten 5 Brände waren in den letzten 10 Jahren
				\item[2019] Waldbrände in Brandenburg, größte Waldbrände der Geschichte des Landes (etwa 1,5 mal größer als bisher größter),
				etwa \SI{7,50}{km\squared} (etwa die Fläche von Dorstfeld) verbrannt.
				\item[2019] Buschfeuer in Australien, \SI{186000}{km\squared} (\SI{20}{\%} der bewaldeten Fläche Australiens, bzw. \nicefrac{1}{2} mal die Fläche von Deutschland) verbrannt, etwa 1 Mrd. Tiere getötet, Rauch auch in Argentinien und Chile spührbar, wird Black Summer genannt.
				\item[2020] zu warmer und trockener Sommer in Deutschland \SI{1,9}{\celsius} über dem Mittel der Referenzperiode 1961 bis 1990, Waldbrände und Hurrikans in den USA, etc.
			\end{itemize}
		}
	\end{frame}

\begin{frame}
	\frametitle{Aktuelle Nachrichten}
	\begin{figure}
		\centering
		\includegraphics[width=0.9\linewidth]{bilder/CurrentSituation_Conclusion}
	\end{figure}
\end{frame}

\begin{frame}
	\frametitle{Aktuelle Nachrichten}
	\begin{figure}
		\centering
		\includegraphics[width=0.7\linewidth]{bilder/dortmund}
	\end{figure}
	\begin{center}
		\textbf{Auch bei uns}
	\end{center}
	\note{
	\begin{itemize}
		\item Man muss gar nicht so weit gehen, in Dortmund ist es genau so (2019)
		\item Diese extremen Ereignisse lassen sich nicht ausschließlich auf den Klimawandel zurück führen. Mit Hilfe von statistischen Modellen ist es aber heute möglich anzugeben, dass sie durch den Klimawandel mit einer viel höheren Wahrscheinlichkeit auftreten.
	\end{itemize}
	}
\end{frame}

% "es war doch immer schon im Sommer warm"


\begin{frame}[t]
	\frametitle{Warming Stripes}
	% Abbildung der Warming Stripes: Die Entwicklung ist wirklich dramatisch

	% Farbskala erklären
	\begin{figure}
		\centering
		\vspace{-1em}
		\begin{tikzpicture}
			\node[anchor=south west,inner sep=0] (image) at (0,0) {
			\includegraphics[width=\linewidth]{bilder/s4f-warming-stripes}};
			\begin{scope}[x={(image.south east)},y={(image.north west)}]
				\only<1>{\draw[thick,->] (0.0, -0.1) -- (1.0, -0.1) node[anchor=north west] {Zeit};}
				% +/- 0.0058 shifts by one year
				\only<2>\fill[gray, opacity=0.7] (0.00,0.003) rectangle (0.3875+0.0058,0.99);
				\only<2>\fill[gray, opacity=0.7] (0.3933+0.0058,0.003) rectangle (1.00,0.99);
				\only<3>\fill[gray, opacity=0.7] (0.00,0.003) rectangle (0.3875-25*0.0058+0.0025,0.99);
				\only<3>\fill[gray, opacity=0.7] (0.3933+5*0.0058+0.0005,0.003) rectangle (1.00,0.99);
				\only<4>\fill[gray, opacity=0.7] (0.00,0.003) rectangle (0.3875+5*0.0058+0.0065,0.99);
				\only<4>\fill[gray, opacity=0.7] (0.3933+36*0.0058-0.0003,0.003) rectangle (1.00,0.99);
				\only<5>\fill[gray, opacity=0.7] (0.00,0.003) rectangle (0.3875+37*0.0058-0.0003,0.99);
				\only<5>\fill[gray, opacity=0.7] (0.3933+67*0.0058-0.002,0.003) rectangle (1.00,0.99);
				\only<6>\fill[gray, opacity=0.7] (0.00,0.003) rectangle (0.3875+68*0.0058-0.002,0.99);
				\only<6>\fill[gray, opacity=0.7] (0.3933+98*0.0058-0.0035,0.003) rectangle (1.00,0.99);
			\end{scope}
	\end{tikzpicture}
		\caption{Die \textit{Warming Stripes}}
		\label{fig:s4f-warming-stripes}
	\end{figure}
	\only<1>{
	\begin{itemize}
		\item Jeder Balken repräsentiert eine Jahr aus der Periode 1850-2017
		\item Blau = kälter als der Mittelwert, Rot=Wärmer
		\item Je dunkeler die Farbe, desto extremer die Abweichung vom Mittelwert
		\item Die Trend geht klar zu höheren Temperaturen
		\item Übrigens auch das Logo der Scientists 4 Future
	\end{itemize}
	\begin{center}
		\url{https://s4f-dortmund.github.io/}
	\end{center}
	}
	\only<3|handout:0>{
		\begin{figure}
			\centering
			\vspace{-1em}
			\includegraphics[trim={0cm 1.5cm 0cm 3cm}, clip, width=.425\textwidth]{bilder/Global_Historical_Climate_Network_Daily/station-counts-1891-1920-temp.png}
			\vspace{-1em}
			\caption{Wetterstationen in der Periode 1891 - 1920 die zehn Jahre lang
							 Wetterdaten übertragen haben, Quelle: Global Historical Climate Network}
		\end{figure}
	}
	\only<4|handout:0>{
		\begin{figure}
			\centering
			\vspace{-1em}
			\includegraphics[trim={0cm 1.5cm 0cm 3cm}, clip, width=.425\textwidth]{bilder/Global_Historical_Climate_Network_Daily/station-counts-1921-1950-temp.png}
			\vspace{-1em}
			\caption{Wetterstationen in der Periode 1921 - 1950 die zehn Jahre lang
							 Wetterdaten übertragen haben, Quelle: Global Historical Climate Network}
		\end{figure}
	}
	\only<5|handout:0>{
		\begin{figure}
			\centering
			\vspace{-1em}
			\includegraphics[trim={0cm 1.5cm 0cm 3cm}, clip, width=.425\textwidth]{bilder/Global_Historical_Climate_Network_Daily/station-counts-1951-1980-temp.png}
			\vspace{-1em}
			\caption{Wetterstationen in der Periode 1951 - 1980 die zehn Jahre lang
							 Wetterdaten übertragen haben, Quelle: Global Historical Climate Network}
		\end{figure}
	}
	\only<6>{
		\begin{figure}
			\centering
			\vspace{-1em}
			\includegraphics[trim={0cm 1.5cm 0cm 3cm}, clip, width=.425\textwidth]{bilder/Global_Historical_Climate_Network_Daily/station-counts-1981-2010-temp.png}
			\vspace{-1em}
			\caption{Wetterstationen in der Periode 1981 - 2010 die zehn Jahre lang
							 Wetterdaten übertragen haben, Quelle: Global Historical Climate Network}
		\end{figure}
	}

	\note<1>{
	\begin{itemize}
		\item[] Warming Stripes sind globale Zusammenfassung für ein Großteil der verfügbaren Temperaturdaten (auch S4F Logo)
		\item[] Jeder Balken Jahresmittelwert globaler Luft- und Wassertemperaturen
		\item[] Bei Interesse: Daten auf unserer Homepage verlinkt (auch für verschiedene Orte verfügbar)
	\end{itemize}
	}
	\note<2>{
	\begin{itemize}
		\item[1916] Letzte große Antarktis-Expedition (Endurance-Expedition) Leitung: Ernest Shackleton. Ziel: Den Kontinent durchqueren. Expedition scheiterte. Besonderes: alle Mitglieder unter widrigsten Umständen überlebt.
		\item[] Albert Einstein veröffentlicht die allgemeine Relativitätstheorie.
		\item[] immer mehr Wetterstationen -> genauere Messungen möglich.
	\end{itemize}
	}
	\note<3->{
	Immer ein 30 Jahrezeitraum hervorgehoben
	\begin{itemize}
		\item[vor 1891] Kaum Wetterstationen, 1-5 Stationen pro Quadrant an US-Küsten, Zentraleuropa und austalischer Ostküste
		\item[ab 1891]	Stationen die zehn Jahre lang täglich Wetterdaten übertrugen, Hier:  Temperatur (auch: Niederschlag)
		\item[] USA 50-100 Stationen pro Quadrant, Europa 1-5 Stationen, Australien 1-10 Stationen an der Küste
		\item[] Restliche Welt vereinzelt Stationen
		\item[ab 1921] Canada und Russland mit 1-5 Stationen abgedeckt
		\item[] Erste Stationen in Afrika
		\item[ab 1951] China, Indien, Argentinien, Großteile von Afrika mit 1-5 Stationen
		\item[] Australien, Europa und Teile Russlands mit 5-10 Stationen
		\item[ab 1981] Abdeckung verbessert, doch große Flächen Brasiliens, Grönlands und der Demokratischen Republik Kongo nicht abgedeck, immer noch Lücken
	\end{itemize}
	}
\end{frame}

\begin{frame}
	\frametitle{Warming Stripes}
	% Extrapolation der Warming Stripes
	\begin{figure}
		\centering
		\includegraphics[width=0.55\linewidth]{bilder/warming_stripes_zukunft}
		\caption{Extrapolation der \textit{Warming Stripes} basierend auf Modellannahmen}
	\end{figure}
	\begin{itemize}
		\item In der Zukunft wird es sehr wahrscheinlich zu einer weiteren Erwärmung kommen
		\item Man kann statistische Prognosen für die Zukunft auf Basis von Modellrechnungen erstellen
		\item Der genaue Trend und die maximale Erwärmung hängen aber davon ab, was wir heute und in den nächsten Jahren tun!
	\end{itemize}

\end{frame}
