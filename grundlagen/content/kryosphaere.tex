\begin{frame}
	\frametitle{Kryosphäre - Eismassen}

	\begin{figure}
		\centering
		\includegraphics{bilder/WMO_Cycles_ice.png}
		\caption{Kryosphäre umfasst alle Formen und Schnee und Eis}
	\end{figure}
	\note{
		\begin{itemize}
			\item[] Eismassen stellen nach den Ozeanen die zweitgrößte Komponente des Klimasystems dar.
			\item[] das liegt an der Wärmekapazität von Wasser und seinem Volumen
      \begin{itemize}
        \item[] Wasser: \SI{4,18}{kJ \per kg \per K} bei \SI{20}{°C}
        \item[] Gestein etwa: \SIrange{0,7}{1}{kJ \per kg \per K}
      \end{itemize}
      \item[] ca. 2\,\% des Wassers ist in fester Form in Gletschern und an den Polkappen dauerhaft gebunden.
  %		\item  inländische Eisflächen speichern ca. 75\% des globalen Süßwassers
  		\item[] Eis reflektiert die ankommende Sonnenstrahlung zu großen Teilen
  		$\rightarrow$ \textbf{\textcolor{blue}{Albedo-Effekt}}\\

  		\item[] Je weniger Eismassen, desto weniger Strahlung wird reflektiert und bleibt in der Atmosphäre $\rightarrow$ Verstärkung des Treibhauseffekts

  		\item[] Durch Rußablagerungen auf den Eismassen wird mehr Sonnenstrahlung absorbiert ("Black Carbon")
  		 $\rightarrow$ Schmelzen des Eis und Verringerung des Albedo-Effekts % (Nature Geoscience, 2014; doi: 10.1038/ngeo2180)
		\end{itemize}
	}
\end{frame}


\begin{frame}
	\frametitle{Kryosphäre - Komponenten}
	Die größten Komponenten der Kryosphäre sind:
	\begin{itemize}
		\item Meereis: schwimmende Eismassen, 19-27 Mio. km$^2$
		\item Eisschilde: über Grönland und der Antarktis, 14 Mio. km$^2$
		\item Permafrost: gefrorene Böden, 22,8 Mio. km$^2$
	\end{itemize}

	\glqq Ein einmal eingesetzter Rückzug großer kontinentaler Eisschilde kann selbst nach Stabilisierung der Randbedingungen [(CO$_2$-Emissionen)] über ein Jahrtausend lang anhalten\grqq{} (Latif, 2009 und IPCC, 2001)\\
	%$\rightarrow$ siehe auch: Trägheit des Klimas in Abbildung \ref{fig:traegheit}

	\note{
		\begin{itemize}
			\item[] Meereis flächenmäßig sehr groß, aber Volumen der Eisschilde deutlich größer
			\item[] Eisschilde durchschnittlich 1,7-2 km dick
			\item[] weitere Elemente:
			\item[] Schneebedeckung variiert jahreszeitlich besingt zwischen 2 und 45 Mio. km$^2$
			\item[] Gletscher bilden 'nur' 0.5 Mio. km$^2$
		\end{itemize}
	}
\end{frame}

\begin{frame}
	\frametitle{Meereis} % Bild -> Pexels Andrea Schettino
  \begin{columns}
    \column{.3\linewidth}
    \begin{figure}
      \centering
      \newlength{\imagewidth}
      \settowidth{\imagewidth}{\includegraphics{bilder/ice-formation-in-body-of-water-3923277.jpg}}
      \includegraphics[trim=0 0 0.69\imagewidth{} 0, clip, width = 0.8\linewidth]{bilder/ice-formation-in-body-of-water-3923277.jpg}
      \caption{Quelle: Pexels, Andrea Schettino}
    \end{figure}
    \column{.7\linewidth}
	    \begin{itemize}
		    \item bildet die Grenze zwischen Ozeanen und Atmosphäre
		    \item Luft über dem Meereis ist deutlich kälter als über dem Ozean
		    \item [$\rightarrow$] Abkühlung der Polarregionen und Verstärkung der atmosphärischen Zirkulation (Winde)
		    \item [$\rightarrow$] beeinflusst Konvektion: je weniger Eis, desto schwächer die Konvektion, desto schwächer sind (langfristig) ozeanische Strömungen
	   \end{itemize}
   \end{columns}

	\note{
		\begin{itemize}
			\item[] Winde entstehen zum Dichte-/Wärmeausgleich, stärkere Tmeperaturunterschiede führen also zu stärkeren Winden
			\item[] Konvektion beeinflusst vorallem die Tiefseestömungen
			\item[] weniger Konvektion bedeutet auch weniger Nährstoffe in oberen Schichten der Ozeane
		\end{itemize}
	}
\end{frame}

\begin{frame}
	\frametitle{Eisschilde}
  \begin{columns}
    \column{.3\linewidth}
    \begin{figure}
      \centering
      \settowidth{\imagewidth}{\includegraphics{bilder/panoramic-view-of-landscape-against-sky-255329.jpg}}
      \includegraphics[trim=0 0 0.69\imagewidth{} 0, clip, width = 0.8\linewidth]{bilder/panoramic-view-of-landscape-against-sky-255329.jpg}
      \caption{Quelle: Pexels, Pixabay}
    \end{figure}
    \column{.7\linewidth}
	\begin{itemize}
		\item Volumen: 25 Mio. km$^3$ (Antarktis) + 3 Mio. km$^3$ (Grönland)
		\item Masse ist durch Niederschlag, Lufttemperatur und Strahlung bestimmt
		\item in der Antarktis gibt es Ausläufer des Landeises auf den Ozean hinaus (Schelfeis) $\rightarrow$ 0,7 Mio. km$^3$
%		\item tiefer liegende Schichten der Eisschilde schmelzen durch den Druck auf die Landmassen $\rightarrow$ erzeugt Spannungen im Eis
		\item Eisschilde schmelzen an randnahen Bereichen
		\item höhere Temperaturen im Nordhalbkugelsommer gefährden das grönländische Eisschild besonders stark
		\item \textbf{Eisschilde haben das größte Potential für einen Anstieg des Meeresspiegels, da sie nicht - wie das Meereis - zum Meeresspiegel beitragen}
	\end{itemize}
  \end{columns}

	\note{
		\begin{itemize}
			\item[] je mehr Schnee und je kälter, desto mehr Eis kann dich bilden
			\item[] Schelfeis zählt nicht zum Meereis, da es fest mit den Eisschilden verbunden ist
			\item[] Schelfeis bildet sich v.a. in Buchten, die von Eisschild umgeben sind
			\item[] Grönland besitzt quasi kein Schelfeis, daher schmilzt in Grönland direkt das Eisschild
			\item[] das Antarktische Eisschild ist durch das Schelfeis etwas mehr geschützt
			\item[] Ein komplettes Abtauen der Eisschilde kann den Meeresspiegel bis zu 64 Meter ansteigen lassen. $\rightarrow$ sog. Meeresspiegel-äquivalent nach IPCC 2007
		\end{itemize}
	}
\end{frame}

\begin{frame}
	\frametitle{Permafrost}
  \begin{columns}
    \column{.3\linewidth}
    \begin{figure}
      \centering
      \settowidth{\imagewidth}{\includegraphics{bilder/Permafrost_stone-rings_hg.jpg}}
      \includegraphics[trim=0 0 0.66\imagewidth{} 0, clip, width = 0.8\linewidth]{bilder/Permafrost_stone-rings_hg.jpg}
      \caption{Quelle: Wikimedia, Hannes Grobe}
    \end{figure}
    \column{.7\linewidth}
	\begin{itemize}
		\item dauerhaft gefrorene Böden (permanent $<$ \SI{0}{\degreeCelsius})
		\item größtenteils in Nordamerika und Eurasien
		\item konserviert unzersetzte organische Materie
		\item [$\rightarrow$] geschätzt 1000 Gt (1 Gt = $10^{15}$ g) Kohlenstoff gespeichert
		\item deutliche Verschiebung der Permafrostgrenze im letzten Jahrhundert beobachtet
		\item die globale Erwärmung gefährdet den Fortbestand von Permafrost
		\item Folgen des Abtauen: Absenken der Böden, Überschwemmungen, Zersetzung bisher konservierter Materie
    \begin{itemize}
      \item[$\rightarrow$] Freisetzung großer Mengen CO$_2$ und Methan
    \end{itemize}
	\end{itemize}
\end{columns}

	\note{
		\begin{itemize}
			\item[] Tiefe des Permafrost ist abhängig von Oberflächen-Temperatur
      \item[] zum Teil mehrere hundert Meter tief gefroren
			\item[] in Sibirien liegt kaum Schnee, sodass die Böden noch kälteren Temperaturen ausgesetzt sind
			\item[] Verschiebung der Permafrost-Grenze in Kanada z.B um 100km nach Norden im letzten Jahrhundert
			\item[] Absinken gefährdet auch lokale Infrastruktur wie Straße, Städte und Öl-Pipelines
			\item[] Überschwemmung gefährdet Wälder $\rightarrow$ 'ertrinken'
      \item[] Wissenschaft nimmt an, dass deutlich mehr Treibhausgase freigesetzt werden als die erwartete neue Vegetation binden würde
		\end{itemize}
	}
\end{frame}

\begin{frame}
	\frametitle{Konsequenzen der Verringerung der Eismassen}
	\begin{itemize}
		\item verringerter Albedo-Effekt
		\item abgeschwächte Konvektion
		\item abgeschwächte Ozeanströmung und Winde
		\item Anstieg des Meeresspiegels
		\item massive Freisetzung von Treibhausgasen aus den Senken Ozean und Permafrost
		\item Trägheit führt zu verzögertem Eintreten der Änderungen
	\end{itemize}

	$\rightarrow$ insgesamt: eine Verstärkung des Treibhauseffekt mit weiteren noch unabsehbaren Folgen

	\note{
		\begin{itemize}
			\item[] Das Schmelzer der Polkappen und Auftauen des Permafrost ist ein deutliches Signal
			\item[] wie gesagt, kann ein einmal in Gang gesetztes Abtauen schwer aufzuhalten sein
			\item[] Die Effekte können deutlich später auftreten
		\end{itemize}
	}
\end{frame}
