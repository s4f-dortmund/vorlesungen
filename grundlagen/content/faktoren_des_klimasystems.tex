\section{Faktoren des Klimasystems}
% Abbildung, die nach und nach wächst, mit den erklärten Teilbereichen

% vgl IPCC AR5 climate feedbacks, AR4 chapter 1 basics: carbon cycle

\begin{frame}
  \frametitle{Faktoren des Klimasystems}
  Atmosphäre \\
  Hydrosphäre = Ozeane und Wasserkreislauf \\
  Kryosphäre = Eismassen \\
  Biospähre = Tiere und Pflanzen \\
  Pedosphäre = Boden \\
  Lithosphäre = Gestein 
  
  \begin{figure}
  	\caption{Schematische Abbildung des Klimasystems}
  \end{figure}
\end{frame}


\begin{frame}
	\frametitle{Hydrosphäre - Wassermassen}
	\begin{itemize}
		% Aufteilung von https://www.lenntech.de/faq-wasser-menge.htm, aber auch Wetzel, Robert G. 1983 in "Periphyton of freshwater ecosystems"
		\item Ozeane - Salzwasser ca. 97\%
		\item Flüsse und Seen - Süßwasser $<$ 1\%
		\item Eismassen ca. 2\% % zu den Eismassen gibt es einen extra Abschnitt
	\end{itemize}
\end{frame}

\begin{frame}
	\frametitle{Hydrosphäre - Wassermassen}
	
	\begin{figure}
		\caption{Schematische Abbildung des Klimasystems, wobei die Wassermassen farblich hervorstechen (z.B. alles transparent außer den Wassermassen)}
		% Inkl Pfeile für Wasserkreislauf
	\end{figure}
\end{frame}


\begin{frame}
	\frametitle{Strahlungseigenschaft des Wassers} % M.Latif Klimawandel und Klimadynamik S. 23
	\begin{itemize}
		\item Wasser ist ein Dipol-Molekül \textcolor{blue}{$H_2O$}
		\begin{itemize}			
			\item<2->[$\rightarrow$] kann viel Wärme aufnehmen bevor es verdampft $\rightarrow$ "Trägheit"
			\item<2->[$\rightarrow$] kann wirksam Infrarotstrahlung absorbieren
		\end{itemize}
		
		%Allgemein liegt die größte Dichte des Wassers bei 4 Grad
		\item<3-> größte Dichte des Wassermoleküls bei 4 $^{\circ} C$ 
		\begin{itemize}
			\item<4->[$\rightarrow$] Eis schwimmt auf Wasser
		\end{itemize}		
		% Im besonderen Fall des Salzwassers liegt die größte Dichte jedoch bei -3.8 Grad
		\item<5-> größte Dichte des Wassermoleküls in Salzwasser -3,8 $^{\circ} C$
		\begin{itemize}
			\item<6->[$\rightarrow$] bis zur Eisbildung sinkt kaltes Wasser in die tieferen Schichten des Ozeans
			\item<6-> [$\rightarrow$] wärmeres, weniger dichtes Wasser steigt auf
			\item<6-> [] \textbf{\textcolor{blue}{Konvektion}} in den Polarregionen 
			\item<6-> [$\rightarrow$] Kohlenstoffsenken
		\end{itemize}
	\end{itemize}

% TODO: evtl. Abbildung der Konvektion: Abgabe von Wärme bei Aufnahme von atmosphärischen Gasen, die dann in die Tiefsee gelangen und dort gespeichert werden

\end{frame}

\begin{frame}
	\frametitle{Wasserdynamik} %M. Latif Klimawandel und Klimadynamik S.24
	\begin{block}{windgetriebene Zirkulation: } % eher horizontal
		Oberflächenströmung der Ozeane durch Reibung, Erdrotation (Corioliskraft) und Form der Meeresbecken 
	\end{block}
	\begin{block}{dichtegetriebene Zirkulation: }  % eher vertikal
		Erwärmung, Abkühlung, und Änderung des Salzgehaltes (durch Eisbildung, Verdunstung oder Niederschlag) haben Einfluss auf die Dichte des Wassers, wodurch die Wassermassen zirkulieren
	\end{block}
	
	% Teil-Abbildung hier einfügen mit Ozean, Flüssen, Sonne, Verdunstung, Wolken und Niederschlag
\end{frame}


\begin{frame}
	\frametitle{Wirkung des Wassers auf das Klima}
	ca. 70\% der Erdoberfläche ist mit Wasser bedeckt\\
	$\rightarrow$ Die \textit{Trägheit} des Wassers ist ein entscheidender Faktor für die Trägheit des Klimas und der Klimaänderungen \\% M.Latif Klimawandel und Klimadynamik S. 23
	Kohlenstoffsenken in der Tiefsee können durch erwärmen der Ozeane *irgendwann* freigesetzt werden $\rightarrow$ massiver Anstieg des atmosphärschen $CO_2 \rightarrow$ Verstarkung des Treibhauseffekts\\
	positive Verstärkung von $CO_2$ und Wasserdampf: eine wärmere Atmosphäre kann mehr ($CO_2$ und) Wasserdampf aufnehmen
\end{frame}

\begin{frame}
	\frametitle{Trägheit des Klimas}
	Die Trägheit des Klimas zeichnet sich dadurch aus, dass Feedbacks verzögert zu einem klimawirksamen Ereignis auftreten. Das heißt, die einzelnen Komponenten des Klimasystems benötigen eine gewisse Zeit sich in den veränderten Bedingungen des Systems zu stabilisieren
	
	
	\begin{figure}
		\caption{Zeitskalen im Klimasystem (Meinecke und Latif, 1995)}
	\end{figure}
	%TODO Abbildung über Verzögerungszeiten/ Zeitskalen abbilden, vgl. M.Latif Klimawandel und Klimasystem S. 14 ursprünglich Meinecke und Latif, 1995
\end{frame}

\begin{frame}
	\frametitle{Kryosphäre - Eismassen}
	
	\begin{figure}
		\caption{Schematische Abbildung des Klimasystems, wobei die Eismassen farblich hervorstechen (z.B. alles transparent außer den Eismassen)}
	\end{figure}
\end{frame}

\begin{frame}
	\frametitle{Kryosphäre - Eismassen} % Eismassen der Erde
	ca. 2\% des Wassers ist in fester Form in Gletschern und an den Polkappen dauerhaft gebunden. \\
	Eis reflektiert die ankommende Sonnenstrahlung zu großen Teilen\\
	$\rightarrow$ \textbf{\textcolor{blue}{Albedo-Effekt}}\\
	
	Je weniger Eismassen, desto weniger Strahlung wird reflektiert und bleibt in der Atmosphäre $\rightarrow$ Verstärkung des Treibhauseffekts
	
	durch Rußablagerungen auf den Eismassen wird mehr Sonnenstrahlung absorbiert $\rightarrow$ Schmelzen des Eis und Verringerung des Albedo-Effekts % (Nature Geoscience, 2014; doi: 10.1038/ngeo2180)
	
\end{frame}

\begin{frame}
	\frametitle{Biosphäre - Tiere und Pflanzen}
	
	\begin{figure}
		\caption{Schematische Abbildung des Klimasystems, wobei die Biosphäre farblich hervorstechen (z.B. alles transparent außer der Biosphäre)}
	\end{figure}
\end{frame}

\begin{frame}
	\frametitle{Biosphäre - Tiere und Pflanzen}
		%TODO
\end{frame}


\begin{frame}
	\frametitle{Kohlenstoffkreislauf}
	
	% nach der Vegetation und Lebewesen
	% --> mit Fokus Fabriken und Ausstoß von CO2
	
	\begin{figure}
		\caption{Schematische Abbildung zum Kohlenstoffkreislauf}
	\end{figure}
	
\end{frame}


\begin{frame}
	\frametitle{Das Klimasystem - Zusammenfassung}
	% Abbildung mit allen zuvor erklärten Elementen
	
	\begin{figure}
		\caption{Abbildung des Klimasystems mit allen zuvor erklärten Elementen}
	\end{figure}
\end{frame}

%Folgende Folie(n) eher nicht einbringen?

%\begin{frame}
%	\frametitle{Atmosphäre}
%	
%\end{frame}

