\section{Faktoren des Klimasystems}
% Abbildung, die nach und nach wächst, mit den erklärten Teilbereichen

% vgl IPCC AR5 climate feedbacks, AR4 chapter 1 basics: carbon cycle

\begin{frame}
  \frametitle{Faktoren des Klimasystems}
  \begin{columns}
	\column{0.7\linewidth}
	\begin{figure}
	 	\centering
	 	\includegraphics[width=\linewidth]{bilder/WMO_Cycles_factors_general.png}
	 	\caption{Schematische Abbildung des Klimasystems, Quelle: IPCC 2007, Kapitel 1 und WMO-Video: The Carbon Cycle} % Anmerkungen hinzugefügt nach
	\end{figure}
  \column{0.3\linewidth}
  \textbf{Sphären des Klimasystems}\\
  (altgriechisch)\\[1em]
  \textit{atmos} = Dunst\\
  \textit{hydor} = Wasser\\
  \textit{kryos} = Eis\\
  \textit{bios} = Leben\\
  \textit{pedon} = Boden\\
  \textit{lithos} = Stein
	%TODO: Evtl Bezeichnungen in Bild einfügen anstatt separat auflisten
	\end{columns}

	\note{
	\begin{itemize}
    \item[] Sphärennamen aus dem altgriechischen
		\item[] Hier: Fokus auf Hydrosphäre, Kryosphäre und Biosphäre, insbesondere Vegetation.
		\item[] Die Effekte in diesen Komponenten sind bereits heute spürbar: Dürreperioden und Überflutungen, Rückgang der Eisschilde und Waldrodung.
		\item[] Die Zustände der Böden sind aber entscheidend für die Vegetation und Landwirtschaft.
		\item[] $\rightarrow$ Erosion von Böden durch Übernutzung und Dürren macht sie unfruchtbar (Nährstoffverlust und Austrocknung).
		\item[] $\rightarrow$ Gestein liefert Mineralien, was ebenfalls für die Vegetation und auch z.B. Schalenbildung bei Muscheln wichtig ist.
	\end{itemize}
	}
\end{frame}


\begin{frame}
	\frametitle{Hydrosphäre - Wassermassen}
	\begin{figure}
		\centering
		\includegraphics{bilder/WMO_Cycles_water.png}
		\caption{Etwa 2/3 der Erdoberfläche sind von Wasser bedeckt, somit ist die Hydrosphäre allein durch die Masse ein wichtiger Faktor des Klimasystems.}
	\end{figure}

	\note{
	\begin{itemize}
		\item[] Etwa 2/3 der Erdoberfläche sind von Wasser bedeckt, somit ist die Hydrosphäre allein durch die Masse ein wichtiger Faktor des Klimasystems.
		\item[] Daher widmen wir uns zuerst den Wassermassen.
		% Aufteilung von https://www.lenntech.de/faq-wasser-menge.htm, aber auch Wetzel, Robert G. 1983 in "Periphyton of freshwater ecosystems"
    \item[] Der Großteil der Wassermassen ist Salzwasser 97\,\%.
    \item[] Süßwasser (Flüsse und Seen) bildet nur weniger als 1\,\% des globalen Wassers.
    \item[] Eismassen machen ca. 2\,\% des Wassers aus und binden den größten Teil des Süßwassers, da das Salz bei der Eisbildung im Wasser gelößt bleibt.
    \item[] Eis betrachten wir später.
	\end{itemize}
	}
\end{frame}


\begin{frame}
	\frametitle{Eigenschaften des Wassers} % M.Latif Klimawandel und Klimadynamik S. 23
	\begin{itemize}
		\item Wasser ist ein Dipol-Molekül \textcolor{blue}{H$_2$O}
		\begin{itemize}
			\item[$\rightarrow$] Kann wirksam Infrarotstrahlung absorbieren
			\item[$\rightarrow$] Kann viel Wärme aufnehmen bevor es verdampft $\rightarrow$ "Trägheit"
		\end{itemize}

		%Allgemein liegt die größte Dichte des Wassers bei 4 Grad
		\item<2-> Größte Dichte von reinem Wasser bei \SI{4}{\degreeCelsius}
		\begin{itemize}
			\item<2->[$\rightarrow$] Eis schwimmt auf Wasser
		\end{itemize}
		% Im besonderen Fall des Salzwassers liegt die größte Dichte jedoch bei -3.8 Grad
		\item<3->Größte Dichte von Salzwasser bei \SI{-3,8}{\degreeCelsius}
		\begin{itemize}
			\item<3-> [] Bei der Eisbildung wird verbleibt das Salz gelößt im Wasser
			\item<3-> [$\rightarrow$] Wasser kann kälter werden als Eis und sinkt in die tieferen Schichten des Ozeans
			\item<3-> [$\rightarrow$] Wärmeres, weniger dichtes Wasser steigt auf
			\item<3-> [] \textbf{\textcolor{blue}{Konvektion}} in den Polarregionen
			\item<3-> [$\rightarrow$] Kohlenstoffsenken
		\end{itemize}
	\end{itemize}

	\note{
		\begin{itemize}
      \item[] erst die Slide vorstellen
			\item[] Das Phänomen Konvektion ist nicht auf Wasser beschränkt. Gibt es auch in der Luft oder wird durch Pumpen erzeugt.
			\item[] die Konvektion ist durch die temperaturbedingte Dichte und den Salzgehalt möglich
			\item[] daher auch unter dem Namen \textit{thermohaline Konvektion} (thermo - Temperatur, halin - Salz) bekannt
		\end{itemize}
	}

% TODO: evtl. Abbildung der Konvektion: Abgabe von Wärme bei Aufnahme von atmosphärischen Gasen, die dann in die Tiefsee gelangen und dort gespeichert werden

%TODO: Erklärung von Senken
\end{frame}

\begin{frame}
	\frametitle{Wasserdynamik} %M. Latif Klimawandel und Klimadynamik S.24
	\begin{figure}
    \centering
    \includegraphics[width=.9\linewidth]{bilder/Thermohaline_Circulation.pdf}
    \caption{Termohaline Zirkulation, Quelle: nach Robert Simmon, NASA}
  \end{figure}

	\note{
		\begin{itemize}
			\item[] Die großen Wassermassen der Ozeane haben eine bestimmte Strömung, z.B. Golfstrom von der Arktis über die Küste Mexikos bis an die Antarktis.
			\item[] Die Stömung ergibt sich vorallem durch die Erdrotation (Corioliskraft).
			\item[] Die oberflächliche Strömung der oberen 100 Meter entsteht durch Wind (und darauf resultierende Reibung) und die Form der Meeresbecken.
			\item[] Die Tiefenströmung wird durch die Konvektion angetrieben. Die Dichte des Wassers spielt dabei eine entschiedende Rolle, da dichteres Wasser nach unten sinkt und leichteres empor steigt.
			\item[] Warme Temperaturen führen zum aufwärmen des Oberflächenwassers und damit zu einer geringeren Dichte.
			\item[] Dadurch werden Oberflächenwasser und Tiefenwasser stark getrennt - und somit strömen die Wassermassen mit unterschiedlicher Geschwindigkeit.
      \item[] Unterscheidung zwischen zwei Zirkulationen
      \begin{itemize}
        \item[Windgetrieben] Oberflächenströmung der Ozeane durch Reibung, Erdrotation (Corioliskraft) und Form der Meeresbecken, eher horizontal
        \item[Dichtegetrieben] Erwärmung, Abkühlung, und Änderung des Salzgehaltes (durch Eisbildung, Verdunstung oder Niederschlag) haben Einfluss auf die Dichte des Wassers, wodurch die Wassermassen zirkulieren, eher vertikal
      \end{itemize}
		\end{itemize}
	}

\end{frame}


\begin{frame}
	\frametitle{Wirkung des Wassers auf das Klima}
	\begin{itemize}
	\item Ca. 70\,\% der Erdoberfläche ist mit Wasser bedeckt
	\item [$\rightarrow$] Die \textit{Trägheit} des Wassers ist ein entscheidender Faktor für die Trägheit des Klimas und der Klimaänderungen % M.Latif Klimawandel und Klimadynamik S. 23
	\item Kohlenstoffsenken in der Tiefsee können durch erwärmen der Ozeane \textit{irgendwann} freigesetzt werden
	\item[$\rightarrow$] massiver Anstieg des atmosphärischen CO$_2$ $\rightarrow$ Verstärkung des Treibhauseffekts
	\item Positive Verstärkung von CO$_2$ und Wasserdampf: eine wärmere Atmosphäre kann mehr Wasserdampf (und CO$_2$) aufnehmen
	\end{itemize}
\end{frame}

\begin{frame}
	\frametitle{Trägheit des Klimas}
	\begin{figure}
		\centering
		\includegraphics[width=0.8\linewidth]{bilder/zeitskala-klimasystem_world_ocean_review.jpg}
		\caption{Zeitskalen im Klimasystem, Quelle: maribus nach Meinecke und Latif, 1995}
		\label{fig:traegheit}
	\end{figure}

  \begin{itemize}
    \item[$\rightarrow$] Verzögertes Feedback bis zu einem klimawirksamen Ereignis.
    \item[$\rightarrow$] Besonders die Ozeane und Eisschilde benötigen eine sehr lange Zeit, um sich geänderten klimatischen Bedingungen anzupassen.
  \end{itemize}

	\note{
		\begin{itemize}
			\item[] Trägheit des Wassers ist entscheident.
			\item[] Die Wassermassen sind in dieser Abbildung stark vertreten.
			\item[] Die untere Atmosphäre passt sich innerhalb weniger Stunden den Bedingungen der Erdoberfläche an (Temeperatur, Gase, etc.)
			\item[] Die Wassermassen reagieren sehr unterschiedlich.
			\item[] Flüsse, Seen und Oberflächenwasser wärmen sich dabei deutlich schneller auf als die tieferen Ozeanschichten. (Das kennt man vielleicht aus Bade- oder Bergseen - die oberen 50 cm sind angenehm warm und darunter liegt deutlich kälteres Wasser)
			\item[] Besonders unterschiedlich schnell reagiert die Biosphäre.
      \begin{itemize}
        \item[] Graslandschaften können schnell austrocken
        \item[] Wälder dagegen verändern sich über Jahrtausende hinweg.
        \item[] Die Vegetation bestimmt in vielen Fällen auch die Ansiedlung von Lebewesen.
        \item[] Die Änderung der Vegetation kann das Ende des Lebenraums einiger Lebewesen bedeuten, aber auch neue Ansiedluneg bedingen.
      \end{itemize}
			\item[] Eine besonders lange Reaktionszeit haben die Eisschilde der Erde. Auf die Eismassen gehen wir als nächstes ein.
		\end{itemize}
	}
\end{frame}

\begin{frame}
	\frametitle{Kryosphäre - Eismassen}

	\begin{figure}
		\centering
		\includegraphics{bilder/WMO_Cycles_ice.png}
		\caption{Kryosphäre umfasst alle Formen und Schnee und Eis}
	\end{figure}
	\note{
		\begin{itemize}
			\item[] Eismassen stellen nach den Ozeanen die zweitgrößte Komponente des Klimasystems dar.
			\item[] das liegt an der Wärmekapazität von Wasser und seinem Volumen
      \begin{itemize}
        \item[] Wasser: \SI{4,18}{kJ \per kg \per K} bei \SI{20}{°C}
        \item[] Gestein etwa: \SIrange{0,7}{1}{kJ \per kg \per K}
      \end{itemize}
      \item[] ca. 2\,\% des Wassers ist in fester Form in Gletschern und an den Polkappen dauerhaft gebunden.
  %		\item  inländische Eisflächen speichern ca. 75\% des globalen Süßwassers
  		\item[] Eis reflektiert die ankommende Sonnenstrahlung zu großen Teilen
  		$\rightarrow$ \textbf{\textcolor{blue}{Albedo-Effekt}}\\

  		\item[] Je weniger Eismassen, desto weniger Strahlung wird reflektiert und bleibt in der Atmosphäre $\rightarrow$ Verstärkung des Treibhauseffekts

  		\item[] Durch Rußablagerungen auf den Eismassen wird mehr Sonnenstrahlung absorbiert ("Black Carbon")
  		 $\rightarrow$ Schmelzen des Eis und Verringerung des Albedo-Effekts % (Nature Geoscience, 2014; doi: 10.1038/ngeo2180)
		\end{itemize}
	}
\end{frame}


\begin{frame}
	\frametitle{Kryosphäre - Komponenten}
	Die größten Komponenten der Kryosphäre sind:
	\begin{itemize}
		\item Meereis: schwimmende Eismassen, 19-27 Mio. km$^2$
		\item Eisschilde: über Grönland und der Antarktis, 14 Mio. km$^2$
		\item Permafrost: gefrorene Böden, 22,8 Mio. km$^2$
	\end{itemize}

	\glqq Ein einmal eingesetzter Rückzug großer kontinentaler Eisschilde kann selbst nach Stabilisierung der Randbedingungen [($CO_2$-Emissionen)] über ein Jahrtausend lang anhalten\grqq{} (Latif, 2009 und IPCC, 2001)\\
	%$\rightarrow$ siehe auch: Trägheit des Klimas in Abbildung \ref{fig:traegheit}

	\note{
		\begin{itemize}
			\item[] Meereis flächenmäßig sehr groß, aber Volumen der Eisschilde deutlich größer
			\item[] Eisschilde durchschnittlich 1,7-2 km dick
			\item[] weitere Elemente:
			\item[] Schneebedeckung variiert jahreszeitlich besingt zwischen 2 und 45 Mio. km$^2$
			\item[] Gletscher bilden 'nur' 0.5 Mio. km$^2$
		\end{itemize}
	}
\end{frame}

\begin{frame}
	\frametitle{Meereis} % Bild -> Pexels Andrea Schettino
  \begin{columns}
    \column{.3\linewidth}
    \begin{figure}
      \centering
      \newlength{\imagewidth}
      \settowidth{\imagewidth}{\includegraphics{bilder/ice-formation-in-body-of-water-3923277.jpg}}
      \includegraphics[trim=0 0 0.69\imagewidth{} 0, clip, width = 0.8\linewidth]{bilder/ice-formation-in-body-of-water-3923277.jpg}
      \caption{Quelle: Pexels, Andrea Schettino}
    \end{figure}
    \column{.7\linewidth}
	    \begin{itemize}
		    \item bildet die Grenze zwischen Ozeanen und Atmosphäre
		    \item Luft über dem Meereis ist deutlich kälter als über dem offenen Ozean
		    \item [$\rightarrow$] Abkühlung der Polarregionen und Verstärkung der atmosphärischen Zirkulation (Winde)
		    \item [$\rightarrow$] beeinflusst Konvektion: je weniger Eis, desto schwächer die Konvektion, desto schwächer sind (langfristig) ozeanische Strömungen
	   \end{itemize}
   \end{columns}

	\note{
		\begin{itemize}
			\item[] Winde entstehen zum Dichte-/Wärmeausgleich, stärkere Tmeperaturunterschiede führen also zu stärkeren Winden
			\item[] Konvektion beeinflusst vorallem die Tiefseestömungen
			\item[] weniger Konvektion bedeutet auch weniger Nährstoffe in oberen Schichten der Ozeane
		\end{itemize}
	}
\end{frame}

\begin{frame}
	\frametitle{Eisschilde}
  \begin{columns}
    \column{.3\linewidth}
    \begin{figure}
      \centering
      \settowidth{\imagewidth}{\includegraphics{bilder/panoramic-view-of-landscape-against-sky-255329.jpg}}
      \includegraphics[trim=0 0 0.69\imagewidth{} 0, clip, width = 0.8\linewidth]{bilder/panoramic-view-of-landscape-against-sky-255329.jpg}
      \caption{Quelle: Pexels, Pixabay}
    \end{figure}
    \column{.7\linewidth}
	\begin{itemize}
		\item Volumen: 25 Mio. km$^3$ (Antarktis) + 3 Mio. km$^3$ (Grönland) = Mio. 28 km$^3$
		\item ihre Masse wird durch Niederschlag, Lufttemperatur und Strahlung bestimmt
		\item in der Antarktis gibt es Ausläufer des Landeises auf den Ozean hinaus (Schelfeis) $\rightarrow$ 0,7 Mio. km$^3$
%		\item tiefer liegende Schichten der Eisschilde schmelzen durch den Druck auf die Landmassen $\rightarrow$ erzeugt Spannungen im Eis
		\item Eisschilde schmelzen an randnahen Bereichen
		\item durch die höheren Temperaturen im Sommer auf der Nordhalbkugel ist das grönländische Eisschild besonders von einer Erwärmung gefährdet
		\item \textbf{Eisschilde haben das größte Potential für einen Anstieg des Meeresspiegels, da sie nicht - wie das Meereis - zum Meeresspiegel beitragen}
	\end{itemize}
  \end{columns}

	\note{
		\begin{itemize}
			\item[] je mehr Schnee und je kälter, desto mehr Eis kann dich bilden
			\item[] Schelfeis zählt nicht zum Meereis, da es fest mit den Eisschilden verbunden ist
			\item[] Schelfeis bildet sich v.a. in Buchten, die von Eisschild umgeben sind
			\item[] Grönland besitzt quasi kein Schelfeis, daher schmilzt in Grönland direkt das Eisschild
			\item[] das Antarktische Eisschild ist durch das Schelfeis etwas mehr geschützt
			\item[] Ein komplettes Abtauen der Eisschilde kann den Meeresspiegel bis zu 64 Meter ansteigen lassen. $\rightarrow$ sog. Meeresspiegel-äquivalent nach IPCC 2007
		\end{itemize}
	}
\end{frame}

\begin{frame}
	\frametitle{Permafrost}
  \begin{columns}
    \column{.3\linewidth}
    \begin{figure}
      \centering
      \settowidth{\imagewidth}{\includegraphics{bilder/Permafrost_stone-rings_hg.jpg}}
      \includegraphics[trim=0 0 0.66\imagewidth{} 0, clip, width = 0.8\linewidth]{bilder/Permafrost_stone-rings_hg.jpg}
      \caption{Quelle: Wikimedia, Hannes Grobe}
    \end{figure}
    \column{.7\linewidth}
	\begin{itemize}
		\item dauerhaft gefrorene Böden durch dauerhaft kalte ($<$ \SI{0}{\degreeCelsius}) Temperaturen
		\item größtenteils in Nordamerika und Eurasien
		\item zum Teil mehrere hundert Meter tief gefroren
		\item Konservierung unzersetzter organischer Materie
		\item [$\rightarrow$] geschätzte Menge an gespeichertem Kohlenstoff: 1000 Gt (1 Gt = $10^{15}$ g)
		\item deutliche Verschiebung der Permafrostgrenze im letzten Jahrhundert beobachtet
		\item die globale Erwärmung gefährdet den Fortbestand von Permafrost
		\item Folgen des Abtauen: Absenken der Böden, Überschwemmungen, Zersetzung bisher konservierter Materie $\rightarrow$ Freisetzung großen Mengen CO$_2$ und Methan
		\item Wissenschaft nimmt an, dass deutlich mehr Treibhausgase freigesetzt werden als die erwartete neue Vegetation binden würde
	\end{itemize}
\end{columns}

	\note{
		\begin{itemize}
			\item[] Tiefe des Permafrost ist abhängig von Oberflächen-Temperatur
			\item[] in Sibirien liegt kaum Schnee, sodass die Böden noch kälteren Temperaturen ausgesetzt sind
			\item[] Verschiebung der Permafrost-Grenze in Kanada z.B um 100km nach Norden im letztn Jahrhundert
			\item[] Absinken gefährdet auch lokale Infrastruktur wie Straße, Städte und Öl-Pipelines
			\item[] Überschwemmung gefährdet Wälder $\rightarrow$ 'ertrinken'
		\end{itemize}
	}
\end{frame}

\begin{frame}
	\frametitle{Konsequenzen der Verringerung der Eismassen}
	\begin{itemize}
		\item verringerter Albedo-Effekt
		\item abgeschwächte Konvektion
		\item abgeschwächte Ozeanströmung und Winde
		\item Anstieg des Meeresspiegels
		\item massive Freisetzung von Treibhausgasen aus den Senken Ozean und Permafrost
		\item Trägheit führt zu verzögertem Eintreten der Änderungen
	\end{itemize}

	$\rightarrow$ insgesamt: eine Verstärkung des Treibhauseffekt mit weiteren noch unabsehbaren Folgen

	\note{
		\begin{itemize}
			\item[] Das Schmelzer der Polkappen und Auftauen des Permafrost ist ein deutliches Signal
			\item[] wie gesagt, kann ein einmal in Gang gesetztes Abtauen schwer aufzuhalten sein
			\item[] Die Effekte können deutlich später auftreten
		\end{itemize}
	}
\end{frame}

\begin{frame}
	\frametitle{Interaktion Hydrosphäre, Kryosphäre und Atmosphäre}
	\begin{figure}
		\centering
		\includegraphics{bilder/WMO_Cycles_factors_waterAndIce.png}
		\caption{Interaktion Hydrosphäre, Kryosphäre und Atmosphäre}
	\end{figure}

	\note{
		\begin{itemize}
			\item[] links an den Pfeilen sind die Wechselwirkungen notiert - z.B. Eis-Ozean-Verbindung (Konvektion)
			\item[] rechts sieht man die allgemeinen Elemente der Komponenten Hydrosphäre und Kryosphäre
			\item[] diese Elemente hängen offensichtlich zusammen und bedingen sich gegenseitig
		\end{itemize}
	}
\end{frame}

\begin{frame}
	\frametitle{Vegetation}

	\begin{figure}
		\centering
		\includegraphics{bilder/WMO_Cycles_land.png}
		\caption{Die Vegetation ist eine interaktive Komponente des Klimasystems}
	\end{figure}

	\note{
		\begin{itemize}
			\item[] Vegetation ist sehr umfangreich und vielseitig
			\item[] steht in direktem Kontakt mit unterer Atmosphäre und Böden
			\item[] durch photosynthetische Prozesse und die Aufnahme sowie Abgabe von Wasser
			\item[] Vegetation hat dadurch massiven Einfluss auf Wetter und Klima - z.B. Tropischer Regenwald, Wolkenbildung
			\item[] zentral ist auch die Rolle beim Stoffkreislauf - u.a. Kohlenstoff, Phosphor, Nitrat, Stickstoff
			\item[] bietet Lebensraum und Lebensgrundlage
		\end{itemize}
	}
\end{frame}

\begin{frame}
	\frametitle{Vegetation}

  \begin{figure}
    \centering
    \includegraphics[width=.55\linewidth]{bilder/Wind_Vegetation.jpg}
    \caption{Wechselwirkung zwischen Atmosphäre und Biosphäre, Quelle: Stadt Kriens}
  \end{figure}
		\begin{itemize}
			\item Bodenbedeckung wirkt auf Wind, Wasseraustausch und Strahlungshaushalt
			\item [$\rightarrow$] Wälder bremsen Winde, speichern Wasser, beeinflussen Wolkenbildung und bilden Schatten und großen Lebensraum
			\item [$\rightarrow$] in Wüsten und Steppen versickert Wasser schneller und es gibt kaum Schatten, dafür existiert schwacher Albedo-Effekt
			\item Existenz und Wachstum von Vegetation bindet u.a. CO$_2$ und absorbiert Strahlung
			\item Absterben von Vegetation führt zu Freisetzung von CO$_2$ und anderen Stoffen in die Luft und Böden % Nitrat, Phosphat, Stickstoff etc.
		\end{itemize}

	\note{
		\begin{itemize}
			\item[] Photosynthese: Aufnahme von CO$_2$ und abgabe von O$_2$, Nutzung des Kohlenstoff für das Wachstum
			\item[] Lösung organischer Kohlenstoff-Verbindungen durch baterielle Zersetzung $\rightarrow$ Freisetzung von CO$_2$
		\end{itemize}
	}
\end{frame}


\begin{frame}
	\frametitle{Interaktion Vegetation und Atmosphäre}

		\begin{figure}
		\centering
		\includegraphics{bilder/WMO_Cycles_factors_landAndGround.png}
		\caption{Interaktion der Biosphäre und Atmosphäre}
	\end{figure}

	\note{
		\begin{itemize}
			\item[] diesmal sind rechts die Wechselwirkungen der Vegetation und Biosphäre zu sehen
			\item[] links sind nochmal die zentralen Elemente der Komponente Vegetation aufgelistet
			\item[] Änderungen an der Landoberfläche durch z.B: Änderung der Landnutzung - Waldrodung, Landwirtschaft...
			\item[] .. ändern auch die Vegetation und die Ökosysteme
		\end{itemize}
	}
\end{frame}



\begin{frame}
	\frametitle{Das Klimasystem - Zusammenfassung}

	\begin{figure}
		\centering
		\includegraphics{bilder/WMO_Cycles_overview.png}
		\caption{Abbildung des Klimasystems mit allen zuvor erklärten Elementen}
	\end{figure}

	\note{
		\begin{itemize}
			\item[] Abbildung mit allen zuvor erklärten Elementen
			\item[] das sind nur die größten Komponenten,
			\item[] viele kleinere Bereiche wie die Böden oder Elemente wie bestimmte Stoffkreisläufe sind ebenfalls wichtig
			\item[] es ist ein sehr komplexes System, an dem in vielen Stellen Änderungen und Wechselwirkungen auftreten
			\item[] also: komplexes System, mit komplexen Zusammenhängen
			\item[] in einzelnen Bereichen wie der Wolkenbildung sind noch Forschungsfragen offen
			\item[] (wir wollen Klarheit reinbringen, damit Lösungen eher im Kontext betrachtet werden können)
			\item[] (Eine Lösung kann nämlich Rückkopplungseffekte in anderen Bereichen erzeugen)
		\end{itemize}
	}
\end{frame}


\begin{frame}
	\frametitle{Kohlenstoffkreislauf}
	\begin{columns}
		\column{0.6\linewidth}
		\begin{figure}
			\centering
			\includegraphics[width=0.9\linewidth]{bilder/IPCC_Cycles_carbon.jpg}
			\caption{Vereinfachte Abbildung zum jährlichen Kohlenstoffkreislauf, Quelle: IPCC 2013, Kapitel 6}
		\end{figure}
		\column{0.4\linewidth}
		$\Rightarrow$ vermutete natürliche CO$_2$-Flüsse vor der Industrialisierung (vor 1750)\\
		\color{red}$\Rightarrow${ }\color{black} durch Menschen verursachte mittlere jährliche CO$_2$-Flüsse (2000 - 2009) \\
		\color{red}\textbf{rote Reservoire}\color{black}: kumulative Änderung zwischen 1750 und 2011
		\begin{center}
			$\downarrow$
		\end{center}
		\textbf{der Kohlenstoffgehalt der Atmosphäre steigt um ca. 4 Gt pro Jahr}
	\end{columns}


	\note{
		\begin{itemize}
			\item[] Die Abbildung ist aus dem neusten IPCC report 2013
			\item[] Sie zeigt die Kohlenstoff-Flüsse schematisch
			\item[] schwarze Pfeile sind die geschätzten jährlichen Emissionen vor der Industrialisierung - also vor 1750
			\item[] rote Pfeile sind die mittleren jährlichen Emissionen zwischen 2000 und 2009
			\item[] Die Kohlenstoff-Speicher in der Erde und der Atmosphäre sind aber die geschätzten \textbf{gesamten Vorkommen}
			\item[] die gesamten Änderungen seit der Industrialisierung sind hierbei wieder \textbf{rot}
			\item[] v.a. aus fossilen Brennstoffen 7,8 Pg C
			\item[] wird im Ozean gespeichert $\rightarrow$ Konvektion

			\item[] \textbf{Fazit: der CO2 Gehalt der Atmosphäre hat sich seit der Industrialisierung beinahe verdoppelt}
			\item[] es ist anzunehmen, dass die Werte heute nochmal höher sind als die angegebenen von 2000-2009
		\end{itemize}
	}
\end{frame}

\begin{frame}
	\frametitle{Kohlenstoffkreislauf - Der Faktor Mensch}
	\begin{itemize}
		\item CO$_2$ ist das wichtigste Spurengas des menschengemachten Treibhauseffekts
		\item es gelangt durch Verbrennung fossiler Brennstoffe vermehrt in die Atmosphäre
		\item ca. 80\% der gestiegenen CO$_2$ Konzentration der Atmosphäre lassen sich auf Verbrennung zurückführen
		\item der restliche Anteil fällt auf die Änderung der Landnutzung, v.a. die Brandrodung
		\item Rückführung auf den Menschen durch Messung von Kohlenstoffisotopen (Anzahl Neutronen in einem Kohlenstoffatom)
		\item[$\rightarrow$] das Verhältnis der Isotope in fossilen Brennstoffen hat durch die Verbrennung Einfluss auf das Verhältnis in der Atmosphäre
	\end{itemize}

	\note{
		\begin{itemize}
			\item[] Isotopmessung:  $^{12}$C hat sechs und $^{13}$C sieben Neutronen, beide haben 6 Protonen, die den Kohlenstoff kennzeichnen
			\item[] durch die Verbrennung entsteht aus organischen $^{12}$C und $^{13}$C CO$_2$ mit entsprechenden Isotopen
			\item[] Isotope werden auch als Masse eines Atoms bezeichnet

			\item[] Übergang nächste Folie: Aber nicht nur der CO$_2$ Gehalt der Atmosphäre hat sich erhöht, sondern auch der von Methan und Lachgas
		\end{itemize}
	}
\end{frame}


\begin{frame}
	\frametitle{Atmosphärisches Methan und Lachgas}
	\begin{figure}
		\begin{columns}
			\column{0.5\linewidth}
				\includegraphics[width=\linewidth]{bilder/IPCC_Cycles_methane.jpg}

			\column{0.5\linewidth}
				\includegraphics[width=1.1\linewidth]{bilder/IPCC_Cycles_n2o_.jpg}
		\end{columns}
		\caption{Vereinfachte Abbildung zum Methan- und Lachgaskreislauf, Quelle: IPCC 2013, Kapitel 6}
	\end{figure}

	\vspace{-1cm}
	Der Gehalt an Methan in der Atmosphäre hat sich zwischen 1750 und 2011 um 250 \,\% erhöht, der Gehalt von Lachgas um etwa 20 \,\%

	\note{
		\begin{itemize}
			\item[] Auch diese Abbildungen sind aus dem IPCC Report von 2013
			\item[] schwarze Pfeile stellen wieder vorindustrielle jährliche Flüsse dar
			\item[] rote Pfeile wieder mittlere jährliche Flüsse zwischen 2000 und 2009
			\item[] Die Speicher/Reservoire sind auch wieder als \textbf{Gesamtwert} angegeben
			\item[] vorindustielles Methan in der Atmosphäre: 1984, heutig: +2970 $\rightarrow$ insgesamt also 4954 Tg Methan
			\item[] Methan: järliches atmosphärisches Plus: 17 (+-9) Tg
			\item[] zusätzliches Methan wird freigesetzt durch:
			\item[] Fossile Brennstoffe: 85-105 Tg
			\item[] Massentierhaltung: 87-94 Tg !!!
			\item[] ganz stark aus Müllhalden (land fills): 67 - 90 Tg
			\item[] $\rightarrow$ durch den anaeroben Abbau der dort gelagerten Stoffe wird von den Mikroorganismen Methan freigesetzt
			\item[] und auch Reisanbau: 33-40 Tg
			\item[] die globalen Speicher von Methan in der Erde sind zusätzlich sehr groß: zB Ozean bis zu 8 Mio Tg Methan

			\item[] vorindustielles Lachgas in der Atmosphäre: 1340, heutig: + 213 $\rightarrow$ insgesamt 1553 Tg N$_2$O
			\item[] Lachgas: järliches atmosphärisches Plus: 3,6 (+-0,15) Tg
			\item[] Lachgas v.a. aus Landwirtschaft (4,1 Tg jährliche Emissionen) Verbrennung von Biomasse und Waldrodung

			\item[] \textbf{Fazit:} durch die starke Treibhauswirksamkeit von Methan und Lachgas, sind diese Entwicklung extrem kritisch zu beobachten!
		\end{itemize}
	}

\end{frame}

\begin{frame}
	\frametitle{Zusammenfassung}
	\begin{itemize}
		\item Klima ist der mittlere Zustand der Atmosphäre an einem bestimmten Ort über einen längeren Zeitraum
		\item Atmosphäre und Strahlungshaushalt der Erde erzeugen lebensfreundliche Bedingungen auf der Erde
		\item Der natürliche Treibhauseffekt entsteht vorallem durch Wasserdampf und CO$_2$.
		\item CO$_2$, CH$_4$ und N$_2$O sind starke Treibhausgase
		\item Durch den Anstieg der atmosphärischen Konzentrationen dieser drei seit der Industrialisierung kann der Mensch als Ursache angenommen werden.
		\item Die Komponenten des Klimasystems hängen stark zusammen, sodass Veränderungen eines Faktors zum Teil deutliche Folgen in anderen Bereichen erzeugen kann.
		\item Manche Änderungen treten erst nach hunderten Jahren auf, manche sind aber auch schon heute und in wenigen Jahren spürbar!
	\end{itemize}
\end{frame}

\begin{frame}
	\frametitle{Zusammenfassung und Ausblick}
	\begin{figure}
		\centering
		\includegraphics[width=\linewidth]{bilder/s4f-warming-stripes}
		\caption{Die Warming Stripes}
	\end{figure}
	$\rightarrow$ Die langfristige Erwärmung und Häufung extremer Wetterereignisse bedeutet eine Veränderung des Klimas!\\
	$\rightarrow$ Viele dieser Änderungen sind in Modellen der Klimaforscher relativ genau vorherzusagen. \\
	$\rightarrow$ Das Ausmaß einiger Änderungen sind aufgrund der langfristigen Effekte nicht konkret vorherzusagen, aber die Tendenz ist klar.
\end{frame}
