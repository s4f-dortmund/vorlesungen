\begin{frame}
	\frametitle{Klimamodelle - Überblick}
	\begin{columns}
	\column{0.65\linewidth}
	\begin{figure}
		\centering
		\includegraphics[width=0.8\linewidth]{%
				bilder/AtmosphericModelSchematic.png}
		\caption{Schematische Abbildung eines Klimamodells, Quelle: NOAA}
	\end{figure}
	\column{0.35\linewidth}
	\begin{itemize}
		\item Klimamodelle bestehen aus Differentialgleichungen
		\item Es gelten Erhaltungssätze (Energie, Impuls, Masse, etc.)
		\item Ströhmungslehre und Chemie werden berücksichtigt.
		\item Berechnet werden unter anderem Wind, Wärmetransfer, Strahlung, Luftfeuchtigkeit und Oberflächenhydrologie
	\end{itemize}
	\end{columns}

	\note{
		\begin{itemize}
			\item[] Klimamodelle sind Computersimulationen, die Differentialgleichungen (Navier-Stokes) nutzen um die chemischen und physikalischen Prozesse des Erdklimas nachzubilden.
			\item[] Das erste Klimamodell enthielt Atmosphäre, Ozean, Landmassen und Seeeis, deckte aber nur \nicefrac{1}{6} der Erdoberfläche ab (Nordpol bis Äquator über einen Winkel von \SI{120}{\degree}).
			\item[] 1956 erstes Modell, dass saisonale Wettermuster reproduzieren konnte.
			\item[$\rightarrow$] general circulation models werden entwickelt.
			\item[$\rightarrow$] Weiterentwicklungen sind Atmospheric (AGCMs) und oceanic GCMs (OGCMs), sowie atmosphere-ocean coupled general circulation models (CGCM or AOGCM).
			\item[] Standard Auflösung eines OGCM (HadOM3): \SI{1.25}{\degree} in Längen- und Breitengrad, mit 20 vertikalen Leveln $\rightarrow$ etwa 1,5 mio. Variablen.
		\end{itemize}
		Typische Parameter eines atmosphärischen Zirkulationsmodells:
		\begin{itemize}
			\item[] Luftdruck, Konvektion, Oberflächenprozesse
			\item[] Albedo, Hydrologie, Wolkenbedeckung
    	\item[] horizontale Geschwindigkeitskomponenten der Atmosphärenschichten
    	\item[] Temperatur und Wasserdampf der Atmosphärenschichten
    	\item[] Solare (Sonne) und Infratotstrahlung (Reflexion der Erde)
		\end{itemize}
	}
\end{frame}

\begin{frame}
	\frametitle{Klimamodelle - im IPCC}
	\begin{figure}
		\begin{columns}
			\column{0.65\linewidth}
				\includegraphics[trim={0cm 8.5cm 0cm 0cm}, clip,width=0.95\linewidth]{bilder/klimamodelle_ipcc_2013.png}
			\column{0.35\linewidth}
				\caption{Verwendete Klimamodelle im IPCC, Quelle: IPCC 2013 Kapitel 9}
		\end{columns}
	\end{figure}

	\note{
	\begin{itemize}
		\item[CMIP5] Coupled Model Intercomparison Project Phase 5
		\item[$\rightarrow$] von CIMP3 nach CIMP5 erhöhte Komplexität und Auflösung, sowie mehr Modelle (41, statt 24)
		\item[AOGCM] Atmosphere–Ocean General Circulation Models
		\item[ESM] Earth System Models
		\item[HT] High-Top Atmosphäre, mit kompletter Stratosphäre und einem Model über der Stratopause
		\item[AMIP] Nur Atmosphäre und Landmassen
		\item[] Genaue Details zu den Modellen im IPCC 2013 Kapitel 9
		\item[] Farbgebung, falls physikalische Gleichung mit wenigstens zwei-wege Kopplung an andere Komponente (erlaubt Klimafeedbacks).
		\item[] Auflösung: hell etwa \num{10000} horizontale Punkte, dunkel etwa \num{100000} horizontale Punkte, bei 20-60 vertikalen Ebenen, also \numrange{200000}{6000000} Punkte mit Variablen für Luftdruck, Temperatur, Wärme, Strahlung, Wasserdampf, etc.
		\item[] Die Auflösung an Land ist typischerweise vergleichbar mit der Auflösung der Atmosphäre.
		\item[] Die Auflösung des Seeeises ist üblicherweise vergleichbar mit der Auflösung des Ozeans.
	\end{itemize}
	}
\end{frame}
