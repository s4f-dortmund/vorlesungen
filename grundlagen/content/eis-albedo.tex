\begin{frame}
	\frametitle{Eis-Albedo-Rückkopplung}%Rahmstorf, Schellnhuber - Der Klimawandel S. 14
  \begin{center}
	 Reflexion der ankommenden Sonneneinstrahlung durch Eismassen\\[3em]
 \end{center}

	 \begin{columns}[T] % align columns
	 	\begin{column}{.48\textwidth}
	 		\centering
	 		\textbf{Abkühlung}\\
	 		\color{blue}\rule{\linewidth}{4pt}
	 		\color{black}
	 		\color<2->{gray}
	 		je mehr Eismassen\\
	 		$\downarrow$\\
	 		desto mehr wird reflektiert/\\
	 		desto weniger wird absorbiert\\
	 		$\downarrow$\\
	 		desto kälter wird es auf dem Planeten\\
	 		$\downarrow$\\
	 		desto weniger Wasserdampf kann die Atmosphäre aufnehmen\\
	 		$\downarrow$\\
	 		desto geringer wird der Treibhauseffekt
	 	\end{column}%
	 	\hfill%
	 	\begin{column}{.48\textwidth}<2->
	 		\centering
	 		\textbf{Aufwärmung}\\
	 		\color{red}\rule{\linewidth}{4pt}
	 		\color{black}
	 		je weniger Eismassen\\
	 		$\downarrow$\\
	 		desto weniger wird reflektiert/\\
	 		desto mehr wird absorbiert\\
	 		$\downarrow$\\
	 		desto wärmer wird es auf dem Planeten\\
	 		$\downarrow$\\
	 		desto mehr Wasserdampf kann die Atmosphäre aufnehmen\\
	 		$\downarrow$\\
	 		desto stärker wird der Treibhauseffekt
	 	\end{column}%
	 \end{columns}
\note{
\begin{itemize}
	\item[] Wechselwirkung unterschiedlicher Faktoren am Beispiel der Eis-Albedo Rückkoppelung
	\item[] Albedo ist im wesentlichen das Reflexionsvermögen eines Körpers auf einer Skala von 0 bis 1, eine hohe Albedo bedeutet viel Reflexion (Eis), eine niedrige Albedo bedeutet wenig Reflexion (Ozean)
	\item[] I.a. gibt es viele ähnliche Rückkoppelungen im Klimasystem, z.B. auch bei Landnutzung/Grünflächen und kann im Prinzip sowohl positiv wie negativ sein
	\item[] Negative Rückkoplung führt zu einem stabilen Verhalten, positive Rückkopplung zu schwer kontrolierbarem Anwachsen.
  \end{itemize}
  }

\end{frame}

% "Runterscrollen" / "Zoom" in einer Abbildung von den Atmosphärischen Schichten auf die Erde
