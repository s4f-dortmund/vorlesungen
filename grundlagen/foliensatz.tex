\documentclass[11pt]{beamer}
\usepackage[utf8]{inputenc}
\usepackage[T1]{fontenc}
\usepackage{lmodern}
\usepackage[ngerman]{babel}
%\usepackage{graphicx}
\usetheme{Singapore}

\author{Scientists for Future - Dortmund}
\title{Physikalische Grundlagen}
%\subtitle{}
%\logo{}
%\institute{}
%\date{}
%\subject{}
%\setbeamercovered{transparent}
%\setbeamertemplate{navigation symbols}{}

\begin{document}
	
	\begin{frame}[plain]
		\maketitle
	\end{frame}

	\begin{frame}
		\frametitle{Übersicht}
		\tableofcontents
	\end{frame}

	
	\section{Wetter und Klima}
	
	\begin{frame}
		\frametitle{Wetter und Klima}
	\end{frame}

	
	\section{Faktoren des Klimasystems}
	
	\begin{frame}
		\frametitle{Faktoren des Klimasystems}
	\end{frame}
	
	\begin{frame}
		\frametitle{Atmosphäre}
	\end{frame}
	
	\begin{frame}
		\frametitle{Ozeane}
	\end{frame}

	\begin{frame}
		\frametitle{Kryosphäre} % Eismassen der Erde
	\end{frame}
	
	\begin{frame}
		\frametitle{Vegetation}
	\end{frame}
	
	
	\section{Strahlungsprozesse der Erde}
	
	\begin{frame}
		\frametitle{Strahlungsprozesse der Erde}
	\end{frame}

	\begin{frame}
		\frametitle{Strahlungshaushalt}
	\end{frame}

	\begin{frame}
		\frametitle{Treibhauseffekt}
	\end{frame}
\end{document}